% !TEX root = ctfp-print.tex

\input{half-title}

\frontmatter
\tableofcontents

\input{content/editor-note}
\chapter*{序言}
\addcontentsline{toc}{chapter}{序言}
\label{序言}
\subfile{content/0.0/preface-zh-cn.tex}

\mainmatter

\part*{第一部分}
\addcontentsline{toc}{part}{第一部分}

\chapter{范畴: 组合的本质}\label{category-the-essence-of-composition}
\subfile{content/1.1/category-the-essence-of-composition-zh-cn}

\chapter{类型和函数}\label{types-and-functions}
\subfile{content/1.2/types-and-functions-zh-cn}

\chapter{小范畴,大范畴}\label{categories-great-and-small}
\subfile{content/1.3/categories-great-and-small-zh-cn}

\chapter{Kleisli 范畴}\label{kleisli-categories}
\subfile{content/1.4/kleisli-categories-zh-cn}

\chapter{积和余积}\label{products-and-coproducts}
\subfile{content/1.5/products-and-coproducts-zh-cn}

\chapter{简单代数数据类型}\label{simple-algebraic-data-types}
\subfile{content/1.6/simple-algebraic-data-types-zh-cn}

\chapter{函子}\label{functors}
\subfile{content/1.7/functors-zh-cn}

\chapter{函子性}\label{functoriality}
\subfile{content/1.8/functoriality-zh-cn}

\chapter{函数类型}\label{function-types}
\subfile{content/1.9/function-types-zh-cn}

\chapter{自然变换}\label{natural-transformations}
\subfile{content/1.10/natural-transformations-zh-cn}

\part*{第二部分}
\addcontentsline{toc}{part}{第二部分}

\chapter{声明式编程}\label{declarative-programming}
\subfile{content/2.1/declarative-programming-zh-cn}

\chapter{Limits and Colimits}\label{limits-and-colimits}
\subfile{content/2.2/limits-and-colimits}

\chapter{Free Monoids}\label{free-monoids}
\subfile{content/2.3/free-monoids}

\chapter{Representable Functors}\label{representable-functors}
\subfile{content/2.4/representable-functors}

\chapter{The Yoneda Lemma}\label{the-yoneda-lemma}
\subfile{content/2.5/the-yoneda-lemma}

\chapter{Yoneda Embedding}\label{yoneda-embedding}
\subfile{content/2.6/yoneda-embedding}

\part*{Part Three}
\addcontentsline{toc}{part}{Part Three}

\chapter{It's All About Morphisms}\label{all-about-morphisms}
\subfile{content/3.1/its-all-about-morphisms}

\chapter{Adjunctions}\label{adjunctions}
\subfile{content/3.2/adjunctions}

\chapter{Free/Forgetful Adjunctions}\label{free-forgetful-adjunctions}
\subfile{content/3.3/free-forgetful-adjunctions}

\chapter{Monads: Programmer's Definition}\label{monads-programmers-definition}
\subfile{content/3.4/monads-programmers-definition}

\chapter{Monads and Effects}\label{monads-and-effects}
\subfile{content/3.5/monads-and-effects}

\chapter{Monads Categorically}\label{monads-categorically}
\subfile{content/3.6/monads-categorically}

\chapter{Comonads}\label{comonads}
\subfile{content/3.7/comonads}

\chapter{F-Algebras}\label{f-algebras}
\subfile{content/3.8/f-algebras}

\chapter{Algebras for Monads}\label{algebras-for-monads}
\subfile{content/3.9/algebras-for-monads}

\chapter{Ends and Coends}\label{ends-and-coends}
\subfile{content/3.10/ends-and-coends}

\chapter{Kan Extensions}\label{kan-extensions}
\subfile{content/3.11/kan-extensions}

\chapter{Enriched Categories}\label{enriched-categories}
\subfile{content/3.12/enriched-categories}

\chapter{Topoi}\label{topoi}
\subfile{content/3.13/topoi}

\chapter{Lawvere Theories}\label{lawvere-theories}
\subfile{content/3.14/lawvere-theories}

\chapter{Monads, Monoids, and Categories}\label{monads-monoids-categories}
\subfile{content/3.15/monads-monoids-and-categories}

\backmatter

\appendix
\addcontentsline{toc}{part}{Appendices}
\input{index}

\makeatletter\@openrightfalse
\chapter*{Acknowledgments}\label{acknowledgments}
\addcontentsline{toc}{chapter}{Acknowledgments}
\input{acknowledgments}

\chapter*{Colophon}\label{colophon}
\addcontentsline{toc}{chapter}{Colophon}
\input{colophon}

\chapter*{Copyleft notice}\label{copyleft}
\addcontentsline{toc}{chapter}{Copyleft notice}
\input{free-software}
\@openrighttrue\makeatother
\afterpage{\blankpage}
