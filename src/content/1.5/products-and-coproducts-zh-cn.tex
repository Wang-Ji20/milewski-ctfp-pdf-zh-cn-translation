% !TEX root = ../../ctfp-print.tex

\lettrine[lhang=0.17]{古} 希腊戏剧家欧里庇得斯曾说过: ``人以群分'', 我们被社会关系所定义.
这在范畴论中尤其正确. 想在范畴中挑出一个物件, 只能通过描述它和其他物件(还有自己)的关系.
这些关系由态射定义.

范畴论中有一个通用的构造方式, 叫\newterm{普遍构造}, 可以用物件之间的关系来定义物件. 一种做法是
挑一个模式, 一个物件和态射构建出的形状, 看看它在范畴里出现多少次. 如果这个模式很常见, 而且范畴很大,
那么很可能有很多匹配. 问题是怎么从中找一个好坏标准, 挑出最好的.

就像上网搜索. 查询就像模式. 一个很宽泛的查询会得到很多结果. 有些相关, 有些不相关. 为了去掉不相关的,
你润色自己的查询关键词, 让它更精确. 最后, 搜索引擎会给结果排序, 希望你想要的结果在最前面.

\section{初始物件}

最简单的形状是单一物件. 显然, 范畴里有多少物件, 就有多少这个形状. 太多了. 我们需要某种排名, 找到排最高的.
手边的工具只有态射可用. 如果把态射看成箭头, 一个物件到另一个物件的箭头净流量可能不同. 这在序集, 例如偏序集中是对的.
我们可以推广这个物件顺序的想法. 如果有一个箭头从物件 $a$ 到物件 $b$, 那么说 $a$ 比 $b$ ``更初始''. 定义
初始物件是那个有箭头到范畴里任何物件的物件. 显然不一定有这个东西, 但没关系. 更大的问题是可能有太多这样的物件:
召回率不错, 但精确率不行. 解决办法要从序范畴里找 --- 它们只允许物件之间最多一个箭头: 一个物件只有一种办法小于等于
另一个物件. 这种做法让我们可以定义初始物件:

\begin{quote}
  有且只有一个箭头到范畴里任何物件的物件是范畴的\textbf{初始物件}.
\end{quote}

\begin{figure}[H]
  \centering
  \includegraphics[width=0.4\textwidth]{images/initial.jpg}
\end{figure}

\noindent
不过, 不保证初始物件是唯一的(如果存在). 但它确保了第二好的事: \newterm{同构唯一性}. 同构在范畴论中很重要, 所以
我很快就会讲到. 现在, 我们只要认可定义里说初始物件 ``是范畴的'' 合理.

例子: 偏序集的初始物件是它的最小元素. 有些偏序集没有初始物件 --- 例如所有整数, 正数和负数, 用小于等于关系作为态射的话.

在集合与函数的范畴里, 初始物件是空集. 还记得, 空集对应 Haskell 的 \code{Void} 类型(在 C++ 里没有对应的类型).
唯一的多态函数从 \code{Void} 到任何其他类型, 它叫 \code{absurd}:

\src{snippet01}
正是这族态射让 \code{Void} 在类型范畴里成为初始物件.

\section{终止物件}

继续说单物件模式, 但稍微改一下排名物件的方法. 我们说物件 $a$ 比物件 $b$ ``更终止'', 如果有箭头从 $b$ 到 $a$.
(注意方向反着来了.) 我们要找一个物件, 它比范畴里任何其他物件都 ``更终止''. 同样, 我们要唯一性:

\begin{quote}
  有且只有一个箭头从范畴里任何物件到该物件的物件是范畴的\textbf{终止物件}.
\end{quote}

\begin{figure}[H]
  \centering
  \includegraphics[width=0.4\textwidth]{images/final.jpg}
\end{figure}

\noindent
同样, 终止物件的唯一性是同构意义上的, 我很快会说. 但先看看例子. 在偏序集里, 终止物件是最大元素(如果存在). 在集合的
范畴里, 终止物件是单例集合. 我们已经讨论过单例集合 --- 它们对应 C++ 的 \code{void} 类型和 Haskell 的 \code{()} 类型.
它只有一个值 --- 在 C++ 中隐式, 在 Haskell 中显式, 用 \code{()} 表示. 我们已经知道, 从任何类型到 \code{()} 的唯一
纯函数是:

\src{snippet02}
所以终止物件的条件都满足了.

留意在这个例子里, 唯一性条件是关键, 因为有一些集合(其实除了空集都行)中有来自每一个集合的态射.
例如, 每个类型都有个布尔函数(谓词):

\src{snippet03}
但是 \code{Bool} 不是终止物件. 至少有一个更多的布尔值函数从每个类型都指向 \code{Bool}
(除了 \code{Void}, 它的两个函数都等于 \code{absurd}):

\src{snippet04}
坚持唯一性给了我们足够的准确度. 准确到让终止物件的定义缩小到只有一个类型.

\section{对偶性}

你当然会发现, 初始物件和终止物件的定义是对称的. 它们之间唯一的区别是态射的方向. 对任意范畴 $\cat{C}$, 我们都可以定义
它的\newterm{opposite category 反范畴} $\cat{C}^{op}$, 只要把所有箭头反过来就行. 反范畴自动满足范畴的所有条件,
只要我们同时重新定义组合. 如果原来态射 $f \Colon a \to b$ 和 $g \Colon b \to c$ 组合成 $h \Colon a \to c$, 而且
$h = g \circ f$, 那么反过来的态射 $f^{op} \Colon b \to a$ 和 $g^{op} \Colon c \to b$ 组合成 $h^{op} \Colon c \to a$,
$h^{op} = f^{op} \circ g^{op}$. 单位态射反过来没变.

对偶性是范畴的重要性质, 因为它把范畴论数学家的产出翻倍了. 你的每个构造都有一个反过来的结果; 证明每个定理, 都附赠一个. 在反范畴里
得到的结论总是加一个 ``余'' 前缀, 所以有积和余积, 单子和余单子, 锥和余锥, 极限和余极限等等. 但没有余余单子, 因为反转两次就回来了.

自然, 终止物件是反范畴的初始物件.

\section{同构}

写程序的都很明白, 定义相等是个不容易的事. 两个对象相等是什么意思? 它们必须占用同一个内存位置(指针相等)? 还是它们所有成员的值都相等?
一个复数用实部和虚部表示, 另一个用模和角表示, 它们相等吗? 你可能觉得数学家一定明白什么是相等, 但他们也不懂. 相等的多重定义也困扰
他们. 有命题相等, 意义相等, 扩展相等, 同伦类型论里的路径相等. 还有更弱的同构, 甚至更弱的等价.

直觉上, 同构的意思是两个对象看起来一样 --- 有相同的形状. 也就是一个物件中的每个部分一一映射另一个物件中的每个部分.
对我们来说, 如果满足这个条件, 两个物件就是一模一样的. 数学上这意味着从物件 $a$ 到物件 $b$ 有一个映射, 从物件 $b$ 到物件 $a$
也有一个映射, 而且它们互为反函数. 在范畴论里, 我们用态射替换映射. 同构就是一对可逆态射, 两个态射互逆.

我们用组合和单位的术语理解逆: 态射 $g$ 是态射 $f$ 的逆, 如果其组合是单位. 实际上有两个等式, 因为有两种组合态射的方法:

\src{snippet05}
我说初始(终止)物件是同构意义上唯一的, 我意思是任意两个初始(终止)物件都是同构的. 一眼就能看出来. 假设有两个初始物件 $i_{1}$ 和
$i_{2}$, 因为 $i_{1}$ 是初始物件, 有唯一态射 $f \Colon i_{1} \to i_{2}$. 同样, 因为 $i_{2}$ 是初始物件, 有唯一态射
$g \Colon i_{2} \to i_{1}$. 这两个态射的组合是什么?

\begin{figure}[H]
  \centering
  \includegraphics[width=0.4\textwidth]{images/uniqueness.jpg}
  \caption{图中所有态射都是独特的.}
\end{figure}

\noindent
组合 $g \circ f$ 必须是从 $i_{1}$ 到 $i_{1}$ 的态射. 但 $i_{1}$ 是初始物件, 所以只有一个态射从 $i_{1}$ 到 $i_{1}$.
因为我们在范畴里, 所以一定存在单位态射从 $i_{1}$ 到 $i_{1}$. 它占了这个位置. 所以 $g \circ f$ 只能等于单位态射. 同样,
$f \circ g$ 也等于单位态射. 这就证明了 $f$ 和 $g$ 是互逆的. 所以任意两个初始物件都是同构的.

注意证明中用到了初始物件到自己态射的唯一性. 没这个我们就证不出 ``同构意义上'' 的部分. 为什么我们需要 $f$ 和 $g$ 唯一呢? 因为
不仅初始物件在同构意义上唯一. 原则上两个物件中间可以有好几个同构关系, 不过不是重点. 这个 ``在唯一同构意义上唯一'' 是所有普遍构造的
重要性质.


\section{乘积}

下一个普遍构造是乘积. 我们知道两个集合的笛卡尔积是什么: 它是对的集合. 但是是什么模式把乘积集合与它的成员集合联系起来了呢?
如果我们能找到这个模式, 我们就能把它推广到其他范畴.

我们现在知道的是, 有两个函数. 投影: 从乘积集合到每个成员集合. 在 Haskell 里, 这两个函数叫 \code{fst} 和 \code{snd}, 它们
从一个对里挑出第一个和第二个成员:

\src{snippet06}

\src{snippet07}
这里, 函数是通过模式匹配参数定义的: 匹配任意对的模式是 \code{(x, y)}, 它把它的成员提取到变量 \code{x} 和 \code{y}.

这些定义可以用通配符简化:

\src{snippet08}
在 C++ 里, 我们会用模板函数, 例如:

\begin{snip}{cpp}
template<class A, class B> A
fst(pair<A, B> const & p) {
    return p.first;
}
\end{snip}
有了这些看着很有限的知识, 试试在集合范畴里定义一个物件与态射的模式. 我们会构造两个集合 $a$ 和 $b$ 的乘积. 这模式里有一个物件 $c$,
还有两个态射 $p$ 和 $q$, 分别把它连接到 $a$ 和 $b$:

\src{snippet09}

\begin{figure}[H]
  \centering
  \includegraphics[width=0.3\textwidth]{images/productpattern.jpg}
\end{figure}

\noindent
所有符合这个模式的 $c$ 都是乘积的候选. 可能有大量的候选.

\begin{figure}[H]
  \centering
  \includegraphics[width=0.4\textwidth]{images/productcandidates.jpg}
\end{figure}

\noindent
例如, 选两个 Haskell 类型做成员. \code{Int} 和 \code{Bool}, 看看它们的乘积的候选有谁.

比如这个: \code{Int}. \code{Int} 可以算是 \code{Int} 和 \code{Bool} 的乘积吗? 可以 --- 它的投影是:

\src{snippet10}
无聊, 但合规.

另一个: \code{(Int, Int, Bool)}. 它是三元组. 这两个态射让它成为合规的候选(我们在三元组上用模式匹配):

\src{snippet11}
第一个候选太小了 --- 它只覆盖了乘积的 \code{Int} 部分; 第二个太大了 --- 它多出了一个 \code{Int} 部分.

但我们还没涉及普遍构造的另一部分: 排名. 我们想比较这个模式下两个不同的实例. 第一个候选物件 $c$ 有两个投影 $p$ 和 $q$,
第二个候选物件 $c'$ 也有两个投影 $p'$ 和 $q'$. 也许只要有一个态射 $m$ 从 $c'$ 到 $c$, 我们就能说 $c$ 比 $c'$ ``更好''
--- 但这太弱了. 我们还想说 $c$ 的投影比 $c'$ 的投影 ``更好'', 或者更 ``普遍''. 意思是 $p'$ 和 $q'$ 的投影可以用
$m$ 从 $p$ 和 $q$ 重建:

\src{snippet12}

\begin{figure}[H]
  \centering
  \includegraphics[width=0.4\textwidth]{images/productranking.jpg}
\end{figure}

\noindent
另一个看这些等式的方式是 $m$ \emph{因子化} $p'$ 和 $q'$. 假装这些等式是在自然数里, 点号是乘法: $m$ 是 $p'$ 和 $q'$ 的公因子.

建立一些直觉. 看看对 \code{(Int, Bool)} 和它的两个典范投影, \code{fst} 和 \code{snd}. 它确实比我刚刚的两个候选都\emph{好}.

\begin{figure}[H]
  \centering
  \includegraphics[width=0.4\textwidth]{images/not-a-product.jpg}
\end{figure}

\noindent
到第一个候选的映射 \code{m} 是:

\src{snippet13}
两个投影, \code{p} 和 \code{q} 可以重建:

\src{snippet14}
第二个例子的 \code{m} 也是同样唯一确定的:

\src{snippet15}
我们可以看到 \code{(Int, Bool)} 比 \code{Int} 和 \code{(Int, Int, Bool)} ``更好''. 那为什么反过来不对?
我们能找到一个 \code{m'} 从 \code{p} 和 \code{q} 重建 \code{fst} 和 \code{snd} 吗?

\src{snippet16}
在第一个例子里, \code{q} 总是返回 \code{True}, 但我们知道有些对的第二个元素会是 \code{False}. 我们不能从 \code{q} 重建
\code{snd}.

第二个例子不一样: 我们运行 \code{p} 或 \code{q} 之后, 仍然保留了足够的信息, 但是有多种方式分解 \code{fst} 和 \code{snd}.
因为 \code{p} 和 \code{q} 都忽略了三元组的第二个成员, 我们的 \code{m'} 可以在第二个成员里放任何东西. 我们可以有:

\src{snippet17}

或者
\src{snippet18}
以此类推.

总之, 给定任意类型 \code{c} 和两个投影 \code{p} 和 \code{q}, 有一个唯一的 \code{m} 从 \code{c} 到因子化
它的乘积 \code{(a, b)}. 其实, 它只是把 \code{p} 和 \code{q} 组合成一个对.

\src{snippet19}
这让笛卡尔积 \code{(a, b)} 成了最优选, 说明这个普遍构造在集合范畴里可用. 它挑出了任意两个集合的积.

忘掉集合, 用相同的普遍构造在任意范畴中定义物件的积. 这样的积不一定存在, 但是如果存在, 它在唯一同构的意义上唯一.

\begin{quote}
  两个物件 $a$ 和 $b$ 的\textbf{积}是物件 $c$ 和两个投影 $p$ 和 $q$, 使得对于任意其他物件 $c'$ 和两个投影,
  有唯一的态射 $m$ 从 $c'$ 到 $c$, 可以因子化那些投影.
\end{quote}

\noindent
从两个候选生产因子化函数的(高阶)函数有时候叫做\newterm{因子化器}. 在我们的例子里, 它是:

\src{snippet20}

\section{余积}

和范畴论里其他构造一样, 积也有一个对偶, 叫做余积. 把积模式里的箭头反过来就得到了 $c$ 和两个\emph{注射}, \code{i} 和 \code{j}:
从 $a$ 和 $b$ 到 $c$ 的态射.

\src{snippet21}

\begin{figure}[H]
  \centering
  \includegraphics[width=0.4\textwidth]{images/coproductpattern.jpg}
\end{figure}

\noindent
排名也反过来了: 物件 $c$ 比物件 $c'$ ``更好'', 当有一个态射 $m$ 从 $c$ 到 $c'$, 可以分解注射 $i'$ 和 $j'$:

\src{snippet22}

\begin{figure}[H]
  \centering
  \includegraphics[width=0.4\textwidth]{images/coproductranking.jpg}
\end{figure}

\noindent
``最好'' 的物件, 有一个唯一的态射把它连接到其他任意模式. 这个物件叫做\newterm{余积}, 若存在, 也是在唯一同构上唯一的.

\begin{quote}
  两个物件 $a$ 和 $b$ 的\textbf{余积}是物件 $c$ 和两个注射, 使得对于任意其他物件 $c'$ 及其两个注射, 存在唯一的态射
  $m$ 从 $c$ 到 $c'$, 可以因子化那些注射.
\end{quote}

\noindent
在集合的范畴里, 余积是集合的\emph{不交并}. $a$ 和 $b$ 不交并中的元素要么来自 $a$, 要么来自 $b$. 如果两个集合有交集, 那么
那些元素在不交并中有两份. 可以把不交并中的元素看成多一个标签, 标记其起源.

对程序员来说, 很容易用类型来理解余积: 它是两个类型的\emph{标签联合}. C++ 支持联合, 但它们没有标签. 这意味着在程序中你必须
记着哪个联合成员是有效的. 要创建一个标签联合, 你必须定义一个标签 --- 一个枚举 --- 并把它和联合结合起来. 例如, 一个
\code{int} 和 \code{char const *} 的标签联合可以这样实现:

\begin{snip}{cpp}
struct Contact {
    enum { isPhone, isEmail } tag;
    union { int phoneNum; char const * emailAddr; };
};
\end{snip}
两个注射可以实现为构造函数, 也可以实现为函数. 例如, 这里第一个注射实现为函数 \code{PhoneNum}:
\code{PhoneNum}:

\begin{snip}{cpp}
Contact PhoneNum(int n) {
    Contact c;
    c.tag = isPhone;
    c.phoneNum = n;
    return c;
}
\end{snip}
它把整数注射进 \code{Contact}.

标签联合也叫做 \newterm{variant}, 在 boost 库中还有一个非常通用的实现, \code{boost::variant}.
\footnote{C++ 17 标准库中已经有了 \code{std::variant}. 在 C++23 还增加了单子操作 --- 译注}

在 Haskell 里, 你可以把把任何数据类型组装成标签联合, 只要用竖线把那些数据构造子分开. \code{Contact} 的例子翻译成:

\src{snippet23}
这里, \code{PhoneNum} 和 \code{EmailAddr} 同时是构造子(注射), 也是模式匹配的标签(之后再提). 例如, 这是用手机号
构造通讯录的方法:

\src{snippet24}
Haskell 乘积的经典实现内置在语言里, 是对. 但是余积的经典实现是一个数据类型, 叫做 \code{Either}, 它在标准库的 \code{Prelude}:

\src{snippet25}
它有两个类型参数, \code{a} 和 \code{b}, 有两个构造子: \code{Left} 和 \code{Right}. \code{Left} 接受一个 \code{a} 的值,
\code{Right} 接受一个 \code{b} 的值.

正如我们定义了乘积的因子化器, 我们也可以定义余积的因子化器. 给定一个候选类型 \code{c} 和两个候选注射 \code{i} 和 \code{j},
\code{Either} 的因子化器产生一个因子函数:

\src{snippet26}

\section{不对称性}

我们已经看过两组对偶的定义: 终止物件的定义可以从初始物件的定义中获得, 只要把箭头的方向反过来就行了;
同样, 余积的定义可以从积的定义中获得. 但是在集合的范畴中, 初始物件和终止物件很不一样. 后面会看到,
积就像乘法, 终止物件是一; 而余积就像加法, 初始物件是零. 特别是对有限集合, 积的大小是每个单体集合大小的乘积,
而余积的大小是每个单体集合大小的和.

这说明在箭头反过来以后, 集合的范畴是不对称的.

注意虽然空集有到任何集合的唯一态射(\code{absurd}函数), 但没有任何态射从其他地方指回来. 单例集合有来自任意集合的唯一态射
, 而且 \emph{还} 有去往任意集合的唯一态射(除了空集). 我们说过, 这些从终止物件出发的态射很重要, 它们挑选其他集合的元素
(空集没有元素,所以没得挑).

正因单例集与积运算的特殊关系, 它才和余积不一样. 考虑用单例集合, 用单元类型 \code{()} 表示, 作为另一个 --- 太差劲的 ---
候选积模式. 给它两个投影 \code{p} 和 \code{q}: 从单元到每个成员集合的函数. 每个函数从它的成员集合选择一个具体元素.
因为积是普遍的, 有一个(唯一)态射 \code{m} 从我们的候选---单例---到积. 这个态射选择一个积集合的元素 --- 它选择一个具体的对.
它还因子化两个投影:

\src{snippet27}
在单例值 \code{()}, 单例集合中唯一的元素上时, 这两个等式变成:

\src{snippet28}
因为 \code{m ()} 是 \code{m} 挑选的积里的元素, 这些方程告诉我们 \code{p} 从第一个集合挑选的元素, \code{p ()},
是 \code{m} 挑选的对的第一个成员. 同样, \code{q ()} 等于第二个成员. 这完全符合我们的理解: 积的元素是成员集合的元素对.

余积就没那么好解释了. 可以试试用单例集合作为候选余积模式, 试试从里面挑选一个元素, 但这就要两个注射进入它, 而不是它得到两个
投影. 它们不会告诉我们其来源(其实, 我们已经看到它们丢弃输入参数). 从余积到单例集合的唯一态射也不会告诉我们什么. 集合这个
范畴从初始物件的方向看起来和从终止物件的方向看起来很不一样.

这不是集合范畴的特性, 而是函数的特性. 我们用函数作为 $Set$ 的态射. 函数总体而言是不对称的. 我解释一下.

函数必须在定义域中每个元素上都有定义(编程中把这样的函数叫做全函数), 但它不必覆盖整个陪域. 极端的例子
是来自单例集合的函数 --- 它们只从陪域选择一个元素. (实际上, 从空集出发的函数更极端.) 当定义域的大小
比陪域小很多的时候, 我们可以把这个函数看成把定义域嵌入到陪域里. 我会叫它\newterm{embedding 嵌入}函数,
但数学家更喜欢命名相反的情况: 填满陪域的函数叫做 \newterm{surjective 满射} 或者 \newterm{onto 映上}.

另一个不对称性是函数可以把定义域中的多个元素映射到陪域中的一个元素. 它们可以折叠定义域.
最极端的例子是把整个定义域映射到单个元素的函数. 对, 多态的 \code{unit} 函数就做这件事.
这种折叠只能通过组合来放大. 两个折叠函数的组合比单体函数更折叠. 数学家把不折叠的函数叫做
\newterm{injective 单射} 或者 \newterm{one-to-one 一对一}.

当然有许多函数不嵌入也不折叠. 这些函数叫做 \newterm{bijections 双射}, 它们是真正对称的, 因为它们是可逆的.
在集合范畴中, 同构和双射是一回事.

\section{挑战}

\begin{enumerate}
  \tightlist
  \item
        说明终止物件在唯一同构上是唯一的.
  \item
        什么是偏序集中两个物件的积? 提示: 用通用构造.
  \item
        什么是偏序集中两个物件的余积?
  \item
        用你最喜欢的编程语言(不能是 Haskell) 实现泛型的 \code{Either} 类型.
  \item
        说明 \code{Either} 比起 \code{int} 和这两个注射是一个更好地余积:

        \begin{snip}{cpp}
int i(int n) { return n; }
int j(bool b) { return b ? 0: 1; }
\end{snip}

        提示: 定义一个函数

        \begin{snip}{cpp}
int m(Either const & e);
\end{snip}

        因子化 \code{i} 和 \code{j}.
  \item
        继续前面的问题: 如何说明 \code{int} 带上两个注射 \code{i} 和 \code{j} 不能比
        \code{Either} 更好?
  \item
        继续: 这些注射呢?

        \begin{snip}{cpp}
int i(int n) {
    if (n < 0) return n;
    return n + 2;
}

int j(bool b) { return b ? 0: 1; }
\end{snip}
  \item
        想一个 \code{int} 和 \code{bool} 余积更差的候选. 它不能比 \code{Either} 更好.
        因为它允许多个可接受的态射从它到 \code{Either}.
\end{enumerate}

\section{参考资料}

\begin{enumerate}
  \tightlist
  \item
        The Catsters,
        \urlref{https://www.youtube.com/watch?v=upCSDIO9pjc}{Products and
          Coproducts} 视频.
\end{enumerate}
