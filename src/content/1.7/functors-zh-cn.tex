% !TEX root = ../../ctfp-print.tex

\lettrine[lhang=0.17]{冒}{着}成为复读机的风险, 我还是要说这个: 函子是一个简单却强大的概念.
范畴论就是这些简单却有力的结论. 函子是一个范畴中间的映射. 对两个范畴 $\cat{C}$ 和 $\cat{D}$,
函子 $F$ 将 $\cat{C}$ 中的物件映射到 $\cat{D}$ 中的物件 --- 它是一个物件的函数. 如果 $a$
是 $\cat{C}$ 中的物件, 我们将它在 $\cat{D}$ 中的像记为 $F a$ (不用括号). 但是范畴不仅有物件
--- 它还有将它们连接起来的态射. 函子也映射态射 --- 它是态射的函数. 但是它不是随意地映射态射
--- 它保留原本的联系. 所以如果范畴 $\cat{C}$ 中的态射 $f$ 将物件 $a$ 连接到物件 $b$,
\[f \Colon a \to b\]
那么 $f$ 在 $\cat{D}$ 中的像 $F f$ 将把 $F a$ 连接到 $F b$:
\[F f \Colon F a \to F b\]

(这个数学和 Haskell 混在一起的表示法希望还算清楚. 把函子应用到物件或者态射时, 我不用括号.)

\begin{figure}[H]
  \centering\includegraphics[width=0.3\textwidth]{images/functor.jpg}
\end{figure}

\noindent
如你所见, 函子保留范畴的结构: 在一个范畴中相连的东西在另一个范畴中也相连. 但是范畴的结构还有
更多的东西: 态射的组合. 如果 $h$ 是 $f$ 和 $g$ 的组合:
\[h = g \circ f\]
我们希望它在 $F$ 下的像是 $f$ 和 $g$ 的像的组合:
\[F h = F g~.~F f\]

\begin{figure}[H]
  \centering
  \includegraphics[width=0.3\textwidth]{images/functorcompos.jpg}
\end{figure}

\noindent
最后, 我们希望 $\cat{C}$ 中的恒等态射映射到 $\cat{D}$ 中的恒等态射:
\[F \idarrow[a] = \idarrow[F a]\]

\noindent
这里, $\idarrow[a]$ 是物件 $a$ 上的恒等态射, $\idarrow[F a]$ 是物件 $F a$ 上的恒等态射.

\begin{figure}[H]
  \centering
  \includegraphics[width=0.3\textwidth]{images/functorid.jpg}
\end{figure}

\noindent

注意这些条件让函子比普通函数更严格. 函子必须保留范畴的结构. 如果你把范畴想象成由态射网连接起来的物件集,
函子就不能把这个结构撕裂. 它可以把物件压缩到一起, 也可以把多个态射粘合成一个, 但它不能把事情搞砸.
这个不撕裂的条件和微积分中的连续性条件很像. 从这个意义上说, 函子是``连续的'' (尽管函子还有更严格的连续性条件).
就像函数一样, 函子可以折叠也可以嵌入. 当源范畴比目标范畴小得多时, 嵌入的方面更突出. 在极端情况下,
源范畴可以是平凡的单例范畴 --- 一个只有一个物件和一个态射 (恒等态射) 的范畴. 从单例范畴到任何其他范畴的函子
只是简单地选择目标范畴中的一个物件. 这和那个性质完全相同: 单例集合到目标集合的态射挑出目标集合中的一个元素.
最大的折叠函子叫做常函子 $\Delta_c$. 它把源范畴中的每个物件映射到目标范畴中的一个选定的物件 $c$. 它也把源范畴中的每个态射
映射到目标范畴中的恒等态射 $\idarrow[c]$. 它就像一个黑洞, 把一切都压缩到一个奇点中. 当我们讨论极限和余极限时, 我们会看到更多
这样的函子.

\section{编程中的函子}

从黑洞回到地球, 继续说编程的事情. 我们有类型与函数的范畴. 我们可以把它们映射到它们自己, 这种函子叫自函子.
所以类型范畴中的自函子是什么? 首先, 它把类型映射到类型. 我们已经见过这映射的例子, 也许还没发觉.
我说的就是被其他类型参数化的类型. 看几个例子.

\subsection{Maybe 函子}

\code{Maybe} 的定义是从类型 \code{a} 到类型 \code{Maybe a} 的映射:

\src{snippet01}
有一个重要的细节: \code{Maybe} 本身不是一个类型, 它是一个\emph{类型构造子}. 你必须给它一个类型, 像
\code{Int} 或者 \code{Bool}, 才能把它变成一个类型. 没有参数的 \code{Maybe} 代表一个类型的函数. 但是
我们可以把 \code{Maybe} 变成一个函子吗? (从现在开始, 当我在编程的语境中说函子时, 我几乎总是指自函子.)
函子不只是物件的映射 (这里是类型), 还是态射的映射 (这里是函数). 对于任何从 \code{a} 到 \code{b} 的函数:

\src{snippet02}
我们想要得到一个函数从 \code{Maybe a} 到 \code{Maybe b}. 为了定义这样的函数, 我们需要考虑两种情况,
对应于 \code{Maybe} 的两个构造子. \code{Nothing} 的情况很简单: 我们只是返回 \code{Nothing}. 如果
参数是 \code{Just}, 我们就把函数 \code{f} 应用到它的内容. 所以 \code{f} 在 \code{Maybe} 下的像是这个函数:

\src{snippet03}
(顺带一提, 在 Haskell 中你可以在变量名中使用撇号', 这在这种情况下非常方便.) 在 Haskell 中, 我们把函子的
态射-映射部分实现为一个叫 \code{fmap} 的高阶函数. 在 \code{Maybe} 的情况下, 它的签名是:

\src{snippet04}

\begin{figure}[H]
  \centering
  \includegraphics[width=0.35\textwidth]{images/functormaybe.jpg}
\end{figure}

\noindent
我们常说 \code{fmap} \emph{lifts 抬升}了一个函数. 抬升的函数作用于 \code{Maybe} 的值. 和往常一样, 因为柯里化,
这个签名可以有两种解释: 单参数函数 --- 参数本身是个函数 \code{(a -> b)} --- 返回一个函数 \code{(Maybe a -> Maybe b)};
或者是一个有两个参数的函数返回 \code{Maybe b}:

\src{snippet05}
基于前面的讨论, 这是我们在 \code{Maybe} 上实现 \code{fmap} 的方式:

\src{snippet06}
为了说明类型构造子 \code{Maybe} 和函数 \code{fmap} 构成一个函子, 我们必须证明 \code{fmap} 保留单位态射和
组合. 这些被称为``函子规则'', 但它们只是在说保留范畴结构而已.

\subsection{等式推理}

为了证明函子规则, 我要使用 \newterm{equation reasoning 等式推理}, 这是 Haskell 中常用的证明技巧. 它利用
了 Haskell 函数的等式定义: 等号的左边等于右边. 你可以相互替换它们, 也许还要重命名变量以避免名字冲突.
把这个想成内联一个函数, 或者相反, 把一个表达式重构成函数. 我们以单位函数为例:

\src{snippet07}
如果你在一些表达式里看到 \code{id y}, 你可以用 \code{y} 替换它 (内联). 进一步, 如果你看到 \code{id} 应用到一个表达式,
比如 \code{id (y + 2)}, 你可以用表达式本身替换它 \code{(y + 2)}. 这种替换是双向的: 你可以用 \code{id e}
替换任何表达式 \code{e} (重构). 如果一个函数是通过模式匹配定义的, 你可以独立地使用每个子定义. 例如, 给定上述的
\code{fmap} 定义, 你可以用 \code{Nothing} 替换 \code{fmap f Nothing}, 或者反过来. 我们看看这有什么用.
从保留单位态射开始:

\src{snippet08}
有两种情况: \code{Nothing} 和 \code{Just}. 我们先看 \code{Nothing} 的情况 (我用 Haskell 伪代码把左边变成右边):

\begin{snip}{haskell}
  fmap id Nothing
= { fmap 的定义 }
  Nothing
= { id 的定义}
  id Nothing
\end{snip}
留意在最后一步我反过来用了 \code{id} 的定义. 我用 \code{id Nothing} 替换了 \code{Nothing}.
实践中, 你用 ``两头凑'' 的方法来完成这种证明 --- 直到你在中间得到同样的表达式 --- 在这里是 \code{Nothing}.
第二种情况也很简单:

\begin{snip}{haskell}
  fmap id (Just x)
= { fmap 的定义 }
  Just (id x)
= { id 的定义 }
  Just x
= { id 的定义 }
  id (Just x)
\end{snip}
下面证明 \code{fmap} 保留组合:

\src{snippet09}
我们先看 \code{Nothing} 的情况:

\begin{snip}{haskell}
  fmap (g . f) Nothing
= { fmap 的定义 }
  Nothing
= { fmap 的定义 }
  fmap g Nothing
= { fmap 的定义 }
  fmap g (fmap f Nothing)
\end{snip}
然后是 \code{Just} 的情况:

\begin{snip}{haskell}
  fmap (g . f) (Just x)
= { fmap 的定义 }
  Just ((g . f) x)
= { 组合的定义 }
  Just (g (f x))
= { fmap 的定义 }
  fmap g (Just (f x))
= { fmap 的定义 }
  fmap g (fmap f (Just x))
= { 组合的定义 }
  (fmap g . fmap f) (Just x)
\end{snip}
强调以下, 等式推理对 C++ 风格有副作用的 ``函数'' 不起作用. 看下面的代码:

\begin{snip}{cpp}
int square(int x) {
    return x * x;
}

int counter() {
    static int c = 0;
    return c++;
}

double y = square(counter());
\end{snip}
用等式推理, 你也许会内联 \code{square} 得到:

\begin{snip}{cpp}
double y = counter() * counter();
\end{snip}
这个转换显然是不对的, 它不会产生相同的结果. 尽管如此, 如果你的 \code{square} 是一个宏, 编译器会尝试使用等式推理,
最后产生不得了的结果.

\subsection{可选值}

函子在 Haskell 中很容易表达, 但在任何支持泛型编程和高阶函数的语言里都可以实现函子. 让我们考虑 \code{Maybe} 的 C++
版本 --- 模板类型 \code{optional}. 这是它的粗略实现 (实际的实现要复杂得多, 因为要处理各种参数传递方式, 拷贝语义,
以及 C++ 中典型的资源管理问题):

\begin{snip}{cpp}
template<class T>
class optional {
    bool _isValid; // the tag
    T _v;
public:
    optional()    : _isValid(false) {}        // Nothing
    optional(T x) : _isValid(true) , _v(x) {} // Just
    bool isValid() const { return _isValid; }
    T val() const { return _v; } };
\end{snip}
这模板提供了函子的一部分定义: 类型的映射. 它把任何类型 \code{T} 映射到一个新类型 \code{optional<T>}.
让我们定义它对函数的作用:

\begin{snip}{cpp}
template<class A, class B>
std::function<optional<B>(optional<A>)>
fmap(std::function<B(A)> f) {
    return [f](optional<A> opt) {
        if (!opt.isValid())
            return optional<B>{};
        else
            return optional<B>{ f(opt.val()) };
    };
}
\end{snip}
这是一个高阶函数, 用函数做参数, 也返回一个函数. 下面是它非柯里化的版本:

\begin{snip}{cpp}
template<class A, class B>
optional<B> fmap(std::function<B(A)> f, optional<A> opt) {
    if (!opt.isValid())
        return optional<B>{};
    else
        return optional<B>{ f(opt.val()) };
}
\end{snip}
还可以让 \code{fmap} 是 \code{optional} 的模板方法. 这么多选择让在 C++ 中抽象函子成为一个问题. 函子应该是一个接口
(不幸的是, 你不能有模板虚函数), 还是一个柯里化或者没柯里化的自由模板函数? 编译器能正确推断缺失的类型吗, 还是必须显式指定?
考虑一种情况, 输入函数 \code{f} 把 \code{int} 映射到 \code{bool}. 编译器怎么推断 \code{g} 的类型:

\begin{snip}{cpp}
auto g = fmap(f);
\end{snip}
特别是如果未来有多个函子重载了 \code{fmap}? (我们很快会看到更多函子.)

\subsection{类型类}

那 Haskell 怎么处理函子的抽象? 它使用类型类机制. 类型类定义了一族支持共同接口的类型. 例如, 支持相等性的对象类定义:

\src{snippet10}
这个定义说, 类型 \code{a} 是 \code{Eq} 类的一员, 如果它支持操作符 \code{(==)}. 这操作符接受两个类型为 \code{a} 的参数,
返回 \code{Bool}. 如果你想告诉 Haskell 一个特定的类型是 \code{Eq}, 你必须声明它是这个类的一个\newterm{instance 实例},
并提供 \code{(==)} 的实现. 例如, 给定 2D \code{Point} 的定义 (两个 \code{Float} 的积):

\src{snippet11}
你可以定义点的相等:

\src{snippet12}
这里我用了操作符 \code{(==)} (我正在定义它) 作为中缀函数, 它在两个模式 \code{(Pt x y)} 和 \code{(Pt x' y')} 之间.
函数体在单等号后面. 一旦 \code{Point} 被声明为 \code{Eq} 的实例, 你就可以直接比较点的相等性. 注意, 和 C++ 或者 Java 不同,
你不必在定义 \code{Point} 时指定 \code{Eq} 类 (或者接口) --- 你可以在客户端代码中这么做. 类型类也是 Haskell 唯一的函数
(或者操作符) 重载机制. 我们需要它来重载不同函子的 \code{fmap}. 有一个复杂的地方: 函子不是一个类型, 而是一个类型构造子.
我们需要一个类型类. 它不能是类型的族, 不能像 \code{Eq} 那样, 应该是一个类型构造子的族. 幸好 Haskell 类型类也适用于
类型构造子. 所以下面是 \code{Functor} 类的定义:

\src{snippet13}
它规定, 如果存在一个函数 \code{fmap} 满足指定的类型签名, 那么 \code{f} 是一个函子. 小写的 \code{f} 是一个类型变量,
类似于 \code{a} 和 \code{b}. 然而, 编译器可以从它的用法推断出它是一个类型构造子而不是一个类型: 它作用在其它类型上,
正如 \code{f a} 和 \code{f b} 那里. 因此, 当声明 \code{Functor} 的实例时, 你必须给它一个类型构造子, 就像 \code{Maybe}:

\src{snippet14}
顺便一提, \code{Functor} 类, 还有它对于很多简单数据类型的实例定义, 包括 \code{Maybe}, 都是标准的 Prelude 库一部分.

\subsection{C++ 中的函子}

我们在 C++ 里可以有同样的做法吗? 类型构造子对应一个模板类, 例如 \code{optional}, 所以同样, 我们会参数化 \code{fmap},
这时它成为一个 \newterm{template template parameter 模板模板参数} \code{F}. 这是它的词法:

\begin{snip}{cpp}
template<template<class> F, class A, class B>
F<B> fmap(std::function<B(A)>, F<A>);
\end{snip}
我们想为不同的函子特化这个模板. 不幸 C++ 禁止部分特化模板函数, 你不能写:

\begin{snip}{cpp}
template<class A, class B>
optional<B> fmap<optional>(std::function<B(A)> f, optional<A> opt)
\end{snip}
所以我们必须退而求其次, 用函数重载, 这让我们回到未柯里化的 \code{fmap} 的定义:

\begin{snip}{cpp}
template<class A, class B>
optional<B> fmap(std::function<B(A)> f, optional<A> opt) {
    if (!opt.isValid())
        return optional<B>{};
    else
        return optional<B>{ f(opt.val()) };
}
\end{snip}
这个定义能用, 只是因为 \code{fmap} 的第二个参数选择了重载对象. 它完全无视了 \code{fmap} 更泛化的定义.

\subsection{列表函子}

为了体会编程中函子的作用, 来看更多的例子. 任何由其它类型参数化的类型都可以是函子. 泛型容器由其存储的类型参数化, 所以
来看看一个非常简单的容器, 列表:

\src{snippet15}
类型构造子 \code{List} 是从任何类型 \code{a} 到类型 \code{List a} 的映射. 为了证明它是一个函子, 我们必须定义
函数的抬升: 给定一个函数 \code{a -> b}, 定义一个函数 \code{List a -> List b}:

\src{snippet16}
在\code{List a} 上作用的函数必须考虑两种情况, 这两种情况对应于两个列表构造子. \code{Nil} 的情况很简单 --- 只返回
\code{Nil} --- 一个空列表没什么可做的. \code{Cons} 的情况有点棘手, 因为它涉及到递归. 所以让我们想想到底怎么做.
我们有一个 \code{a} 类型的列表, 一个函数 \code{f} 把 \code{a} 映射到 \code{b}, 我们想生成一个 \code{b} 类型的列表.
显然我们得用 \code{f} 把列表的每个元素从 \code{a} 映射到 \code{b}. 在这个例子里, 列表定义为头和尾的 \code{Cons},
那又怎么做呢? 我们可以把 \code{f} 应用到头上, 然后把抬升的 (\code{fmap}过的) \code{f} 应用到尾上. 这是一个递归定义,
因为我们在定义抬升的 \code{f} 时用到了抬升的 \code{f}:

\src{snippet17}

注意在右手边, \code{fmap f} 应用到一个比定义时更短的列表上 --- 它应用到了列表的尾上. 我们递归到越来越短的列表, 所以
最后一定回到空列表, 或者说 \code{Nil}. 但我们之前已经说过了 \code{fmap f} 在 \code{Nil} 上返回 \code{Nil}, 所以
结束递归. 为得到最终结果, 我们组合新的头 \code{(f x)} 和新的尾 \code{(fmap f t)}, 用 \code{Cons} 构造一个新的列表.
综上所述, 这是列表函子的实例声明:

\src{snippet18}
如果你更习惯 C++, 考虑 \code{std::vector}, 这个 C++ 里最泛用的容器. 它的 \code{fmap} 实现只是 \code{std::transform}
很薄的一层封装.

\begin{snip}{cpp}
template<class A, class B>
std::vector<B> fmap(std::function<B(A)> f, std::vector<A> v) {
    std::vector<B> w;
    std::transform( std::begin(v)
                  , std::end(v)
                  , std::back_inserter(w)
                  , f);
    return w;
}
\end{snip}
我们可以用它来平方一个数字序列:

\begin{snip}{cpp}
std::vector<int> v{ 1, 2, 3, 4 };
auto w = fmap([](int i) { return i*i; }, v);
std::copy( std::begin(w)
         , std::end(w)
         , std::ostream_iterator(std::cout, ", "));
\end{snip}
许多 C++ 容器都是函子, 因为它们实现了迭代器, 这些迭代起可以传给 \code{std::transform}, 这就是 \code{fmap} 的表亲.
不巧, 函子丢失了其精简, 迷失在一堆迭代器和模板的藻饰中(看看前面的 \code{fmap} 实现). 我很高兴地说, 新提出的 C++ range
库让 range 的函数式本性更突出了.

\subsection{读者函子}

现在你已经多少有了点感觉 --- 例如,函子可以是某种容器 --- 下面我要介绍一些第一眼看上去很不一样的东西. 考虑一个类型\code{a}映射到
一个返回 \code{a} 的函数. 我们还没有深入讨论函数类型 --- 完整的范畴论解释快来了 --- 但是我们对它们有一些程序员通常的理解.
在 Haskell 中, 函数类型是用箭头类型构造子 \code{(->)} 构造的, 它接受两个类型参数: 参数类型和结果类型. 你已经在中缀形式
\code{a -> b} 中见过它, 但是它也可以用前缀形式, 只要括起来:

\src{snippet19}
就像普通的函数, 超过一个参数的类型函数也可以部分应用. 所以当我们只给箭头一个类型参数时, 它还需要另一个. 这就是为什么:

\src{snippet20}
是一个类型构造子. 它需要另一个类型 \code{b} 才能构造出一个完整的类型 \code{a -> b}. 现在, 它只是定义了一个完整的类型
构造子族, 这个族的参数是 \code{a}. 我们看看这是不是一族函子. 处理两个类型参数可能有点困难, 所以来重命名一下. 我们把
参数类型叫 \code{r}, 结果类型叫 \code{a}, 和之前的函子定义一致. 所以我们的类型构造子把任何类型 \code{a} 映射到类型
\code{r -> a}. 为了证明它是函子, 我们想把函数 \code{a -> b} 抬升到函数接受 \code{r -> a} 返回 \code{r -> b}.
这些是用类型构造子 \code{(->) r} 作用在 \code{a} 和 \code{b} 上构造的类型. 这是 \code{fmap} 应用到这个情况的类型签名:

\src{snippet21}
我们要解决这个难题: 给一个函数 \code{f :: a -> b} 和一个函数 \code{g :: r -> a}, 构造一个函数 \code{r -> b}.
只有一种方法组合这两个函数, 这结果正是我们想要的. 所以这就是 \code{fmap} 的实现:

\src{snippet22}
成功! 如果你喜欢简洁的符号, 这个定义可以进一步简化, 因为组合可以用前缀形式重写:

\src{snippet23}
参数可以省略, 来得到两个函数的直接等式:

\src{snippet24}
这个类型构造子 \code{(->) r} 和上面的 \code{fmap} 实现的组合叫做读者函子.

\section{函子当成容器}

我们已经看过一些编程语言中函子的例子. 它们定义了通用的容器, 或者至少是一些包含某种类型的对象的容器. 读者函子看上去是个例外,
因为我们一般认为函数不是数据. 但我们看到纯函数可以被备忘下来, 函数执行因而成为查表. 表就是数据. 反过来, 因为 Haskell 的
惰性求值, 传统的容器, 比如列表, 或许实际上实现为函数. 比如考虑一个自然数的无穷列表, 它可以这样定义:

\src{snippet25}
第一行, 一对方括号是 Haskell 内建的列表类型构造子. 第二行, 方括号用来创建一个列表字面量. 很明显, 这样的无穷列表不能
存在内存里. 编译器把它实现成一个函数, 只在需要的时候生成 \code{Integer}. Haskell 模糊了数据和代码的界限. 列表
可以看成函数, 函数可以看成一张表, 把参数映射到结果. 后者甚至很实用, 如果函数定义域有限而且不大. 不过就不太可能把
\code{strlen} 实现成查表了, 因为字符串有无穷多个. 作为程序员, 我们不喜欢无穷, 但是在范畴论里, 你天天见到无穷.
不论是所有字符串的集合, 还是所有宇宙的状态, 包括过去, 现在, 和未来 --- 我们都能处理! 所以我喜欢把函子对象 (一个
由自函子生成的对象) 看成包含了它参数化的类型的值或值的集合, 即使这些值并不真的存在. 一个函子的例子是 C++ 的
\code{std::future}, 它可能包含一个值, 但不保证会有; 如果你想访问它, 你可能会阻塞等待另一个线程执行完毕.
另一个例子是 Haskell 的 \code{IO} 对象, 它可能包含用户输入, 或者未来宇宙的某个版本, 那时候显示器上显示着
 ``Hello World!''. 按照这个解读, 函子对象这东西或许有一个或多个参数化类型的值. 或者它可能包含生成这些值的
 方法. 我们其实不关心如何访问那些值 --- 它完全是可选的, 而且完全在函子的范围之外. 我们只关心如何用函数操作
 那些值. 如果那些值可以访问, 我们就能看到操作的结果. 如果不能, 我们就只能关注操作的组合是否正确, 还有确保
 单位函数不改变任何东西. 为了说一下我们多不关心访问函子对象中的值, 我写了一个类型构造子, 完全忽略其参数 \code{a}:

\src{snippet26}
\code{Const} 类型构造子接受两个类型, \code{c} 和 \code{a}. 和之前用箭头构造子一样, 我们打算部分应用它,
来新建一个函子. 这个数据构造子 (也叫 \code{Const}) 只接受一个 \code{c} 类型的值. 它和 \code{a} 没有关系.
这构造子的 \code{fmap} 类型是:

\src{snippet27}
因为这函子忽略其类型参数, \code{fmap} 就能随意忽略它的函数参数 --- 函数没有东西可以操作:

\src{snippet28}
这在 C++ 可能更清晰 (未曾设想的道路), 因为那里类型参数和值有更强的区分 --- 类型参数是编译时的, 值是运行时的:

\begin{snip}{cpp}
template<class C, class A>
struct Const {
    Const(C v) : _v(v) {}
    C _v;
};
\end{snip}
C++ 的 \code{fmap} 实现也忽略掉函数参数, 本质上就是转换一下 \code{Const} 的类型, 而不改变值:

\begin{snip}{cpp}
template<class C, class A, class B>
Const<C, B> fmap(std::function<B(A)> f, Const<C, A> c) {
    return Const<C, B>{c._v};
}
\end{snip}
虽然很怪, 但是 \code{Const} 函子在许多构造中都很重要. 在范畴论里, 这是我之前提到 $\Delta_c$ 函子的特例 ---
那个黑洞的自函子例子. 我们将来会看到更多它的用法.

\section{函子组合}

不难说服你: 范畴之间的函子是可以组合起来的, 就像集合中的函数可以组合. 函子的组合, 作用在物件上的时候, 就只是
它们各自物件映射的组合; 作用在态射上的时候也类似. 通过两个函子以后, 单位态射应该还是单位态射, 态射的组合也还是
态射的组合. 这没什么. 组装自函子尤其容易. 记得函数 \code{maybeTail} 吗? 用 Haskell 内建的列表实现重写:

\src{snippet29}
(空的方括号对\code{{[}{]}} 替换了我们之前用来调用 \code{Nil} 的空列表构造子. \code{Cons} 构造子被中缀操作符
\code{:} 替换了.) \code{maybeTail} 的结果类型是两个函子的组合, \code{Maybe} 和 \code{{[}{]}}, 作用在 \code{a} 上.
每个函子都有自己的 \code{fmap}, 但是如果我们想把某个函数 \code{f} 应用到这个组合(Maybe 的列表)的内容, 怎么办? 我们必须穿过
两层函子. 我们可以用 \code{fmap} 穿过外层的 \code{Maybe}. 但是我们不能直接把 \code{f} 传给 \code{Maybe}, 因为
\code{f} 不作用在列表上. 我们必须把 \code{(fmap f)} 传给内层的列表. 例如, 看看怎么把 \code{Maybe} 的整数列表
平方:

\src{snippet30}
编译器在分析类型之后会发现, 外层的 \code{fmap} 要使用 \code{Maybe} 的那个实现, 内层的 \code{fmap} 要使用
列表函子的实现. 也许不是很显然, 但是上面的代码可以重写成:

\src{snippet31}
但是记得 \code{fmap} 可以当成只有一个参数的函数:

\src{snippet32}
在我们这里, \code{(fmap . fmap)} 的第二个 \code{fmap} 接受:

\src{snippet33}
返回这个类型的函数:

\src{snippet34}
第一个 \code{fmap} 接受这个函数, 返回另一个函数:

\src{snippet35}
最后, 这个函数应用在 \code{mis} 上. 所以两个函子的组合, 产生一个新的函子. 这个函子的 \code{fmap} 就是原来两个函子
的 \code{fmap} 组合起来. 回到范畴论: 显然函子组合符合结合律(物件的映射是结合的, 态射的映射也是结合的). 每个范畴里
也都有一个平凡的单位函子: 它把每个物件映射到它自己, 把每个态射映射到它自己. 所以函子的性质很像范畴中的态射, 那这是
哪个范畴呢? 这个范畴里, 物件应该是范畴, 态射应该是函子. 这是范畴之范畴. 但一个含有 \emph{所有} 范畴的范畴应该包括自己,
我们不会进去那个让集合的集合不存在的悖论. 不过有一个范畴包含所有 \emph{小} 范畴, 叫做 $\Cat$(但是这个范畴自己很大,
所以不包括自己). 小范畴的所有物件可以放到一个集合里, 这范畴里没有比集合更大的东西. 记着, 在范畴论里, 无限不可数的集合都是
``小'' 的. 我说这些, 是因为我发现很奇妙: 我们可以在好几层抽象以后, 还看到相同的结构重复自己. 我们之后会看到这些
函子同样形成范畴.

\section{挑战}

\begin{enumerate}
  \tightlist
  \item
        我们把 \code{Maybe} 类型构造子像这样定义:

        \begin{snip}{haskell}
fmap _ _ = Nothing
\end{snip}

        忽略了所有的参数. 这还是函子吗? (提示: 检查函子规则)
  \item
        证明读者函子符合函子规则. 提示: 很简单.
  \item
        用你第二喜欢的语言实现读者函子.(第一喜欢的当然是 Haskell)
  \item
        证明列表函子符合函子规则. 假设这些规则对列表的尾部成立(用\emph{归纳法}证明).
\end{enumerate}
