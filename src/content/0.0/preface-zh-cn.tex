% !TEX root = ../../ctfp-print.tex

\begin{quote}
  很久以来我一直想写一本书, 讲的是范畴论, 讲给程序员听. 注意, 不是计算机科学家, 而是程序员——
  他们是工程师而不是科学家. 听上去很吓人, 我也怕写不好. 科学和工程区别很大, 我当然知道,
  因为我在两边都工作过. 但我很想把事情解释清楚. 我崇拜理查德·费曼\(Richard\ Feynman\),
  他是大师, 简单几句就能把事情讲明白. 我知道我不是费曼, 不过我要努力. 开始之际, 我
  写这篇序言来鼓励读者学习范畴论, 希望引起讨论并激发反馈. \footnote{
    你也许想看看我的视频课程, 在
    \href{https://goo.gl/GT2UWU}{https://goo.gl/GT2UWU} (或者在 Youtube 搜索
    ``bartosz milewski category theory'')}
\end{quote}

\lettrine[lhang=0.17]{我}{要} 用几个段落说服你:这书是写给你的. 学习数学中最
抽象的领域要花费 ``大量空闲时间'', 你也许会这样反对我, 但你的担忧毫无根据.

我那么乐观, 基于以下原因. 首先, 范畴论是一座宝库, 其中含有非常实用的编程思想. Haskell
程序员已经利用这些资源很久了, 这些思想也渗透到其他编程语言里, 但渗透得很慢. 我们需要
加快速度.

第二, 有很多种数学, 它们吸引不同的人. 你也许讨厌微积分或线性代数, 但你不见得讨厌范畴论.
我要说, 范畴论是最适合程序员的数学. 这是因为范畴论研究的对象不是具体的东西, 而是结构;
这种结构让程序可以组合.

组合是范畴论的精髓, 范畴的定义中就含有组合. 我要说组合就是编程的本质. 我们
一直在组装东西, 先驱还没有发明子过程时, 我们就在组合. 很久以前, 结构化编程革新了编程,
因为它们让代码分成块, 这些块相互组合. 然后面向对象出现, 它组合对象. 函数式编程
不仅能组合函数和代数数据结构, --- 它还可以组合并发 ---, 这个功能其他编程范式都做不了.

第三, 我有一件秘密武器, 一把屠刀. 我会把数学砍到能被程序员理解. 如果你是一名数学家, 你
要非常小心, 弄对每一个假设, 准确限定每一个语句, 严格构造证明. 所以门外汉很难读懂数学书和
数学论文. 我是学物理的, 在物理中我们广泛应用非形式推理. 数学家嘲笑狄拉克 delta 函数, 它
由伟大的物理学家 P.A.M Dirac 用来解一些微分方程. 他们后来笑不出来了, 因为发现了一门新的
微积分分支——分布理论, 它形式化了 Dirac 的洞见.

不过用通俗的话讲数学, 有可能完全讲错了;所以我会确保书中所有的非正式结论背后都有严格的数学基础.
我真有一本 Saunders Mac Lane 的 \emph{Categories for the Working Mathematician}
在床头柜上.

因为这是 \emph{给程序员} 的范畴论, 我会用代码解释所有的重要概念. 你也许会发现, 比起常用的
命令式编程语言, 函数式编程语言更像数学一些. 函数式语言还有更强的抽象本领. 所以自然会开始想:
也许要先学习 Haskell 才能学明白范畴论. 这意味着范畴论只能用在函数式编程里, 所以是错的.
我会写大量的 C++ 例子. 你一定会看到一些丑陋的语法, 代码不会非常可读;你也许要多次复制粘贴,
来弥补 C++ 缺少的高层抽象能力, 但这也不过是普通 C++ 程序员每天都在做的事情.

提到 Haskell 的时候, 你也不必茫然无措. 你不需要成为 Haskell 程序员, 你只需要一门语言, 用来
勾画思想, 便于你在 C++ 中实现这些想法. 这其实也是我当初学习 Haskell 的契机. 我发现它语法精简,
类型系统很强大, 非常适合用来理解和实现 C++ 模板、数据结构和算法. 但我不能指望人人学过 Haskell,
所以我会在这本书中慢慢介绍一切.

如果你是老程序员, 你一定扪心自问:我已经写这么久代码了, 从来没见过范畴论或者函数式, 世界变了?
当然, 你惯用的命令式语言中已经侵入了大量的函数式特性. 甚至 lambda 表达式都闯入了 OOP 最
坚实的堡垒, Java 语言中. C++ 语言这几年同样飞速演化 --- 几年一个新标准 ---, 想要跟上
时代. 所有这些都在导向一个突变, 或者我们学物理的叫做相变. 煮热水总会沸腾. 我们正是一只在温水
里的青蛙, 需要决定是继续呆在水里, 还是离开呢?

\begin{figure}[H]
  \centering
  \includegraphics[width=0.5\textwidth]{images/img_1299.jpg}
\end{figure}

\noindent

巨变之一是多核革命. 流行的 OOP 在并发和并行编程领域毫无作用, 反而容易产出危险和漏洞百出
的设计. 数据隐藏是 OO 的基本前提, 但它只要加上共享和互斥, 就成为数据竞争的温床. 用
互斥锁保护代码看上去很好, 但是锁不能组合, 而且把锁藏起来的话, 更容易死锁, 而且更难 Debug.

即使不说并发, 软件系统越来越复杂, 其规模效应也让命令式编程范式无法应付. 简单说, 副作用
已经控制不住了. 确实, 有副作用的函数很方便, 写起来也很容易. 他们的效果写在名字和注释里.
叫做 SetPassword 或 WriteFile 的函数显然改变一些状态, 并产生一些副作用, 我们习惯
处理这些. 但组合这些函数之后, 副作用发生在有副作用的函数之上, 情况就变得极其
复杂. 不是说副作用绝对不好, 只是它们隐藏起来, 所以在规模很大的时候没有办法管理. 副作用
不能规模化, 而命令式编程基本都是副作用.

硬件改变, 软件越来越复杂, 所以我们要重新思考编程的基础. 就像那些建造欧洲伟大的哥特教堂
的建筑师, 我们磨练自己的工艺, 直到材料和结构的极限. 有一个未完成的哥特教堂矗立在法国,
\urlref{http://en.wikipedia.org/wiki/Beauvais_Cathedral}{Beauvais},
凝视着人们追求极致的努力. 它本来要用来打破所有之前的高度和亮度记录, 但倒塌了好几次.
临时措施, 例如铁棍和木撑虽然让它免于解体, 但无疑很多东西大错特错. 从现代观点看, 那么多
哥特建筑建成, 完全不借助现代材料科学、计算机建模、有限元分析、数学、物理, 堪称奇迹.
我希望未来的人们看到我们在操作系统、网络服务器、互联网上展现的编程技巧, 会同样惊讶.
而且, 他们理应惊讶, 因为我们做这些都只在脆弱的理论基础上. 如果我们要前进就必须巩固
这些基础.

\begin{figure}
  \centering
  \includegraphics[totalheight=0.5\textheight]{images/beauvais_interior_supports.jpg}
  \caption{临时措施, 用来避免 Beauvais 大教堂倒塌. }
\end{figure}
