% !TEX root = ../../ctfp-print.tex

\lettrine[lhang=0.17]{既}{然}我们已经知道什么是函子, 也看了一些例子. 接下来看看如何用小函子建造大函子.
看看什么类型构造子(对应于范畴中物件之间的映射)可以扩展为函子(包括态射之间的映射)是格外有趣的.

\section{双函子}

既然函子是 $\Cat$ (范畴之范畴)中的态射, 大量态射的直觉 --- 尤其是函数 --- 自然也适用于函子. 例如, 正如可以有
两个参数的函数, 也可以有两个参数的函子, 或者说一个 \newterm{bifunctor 双函子}. 在物件上, 双函子把每一对物件,
一个来自范畴 $\cat{C}$, 一个来自范畴 $\cat{D}$, 映射到范畴 $\cat{E}$ 中的一个物件. 注意这就是在说它只是从
范畴的\newterm{Cartesian product 笛卡尔积} $\cat{C}\times{}\cat{D}$ 到 $\cat{E}$ 的映射.

\begin{figure}[H]
  \centering\includegraphics[width=0.3\textwidth]{images/bifunctor.jpg}
\end{figure}

\noindent
这很直白. 但是函子性意味着双函子也要映射态射. 不过这时它就需要映射一对态射了, 一个来自 $\cat{C}$, 一个来自
$\cat{D}$, 把它们映射到 $\cat{E}$ 中的一个态射.

依然, 一对态射只是积范畴 $\cat{C}\times{}\cat{D}$ 到 $\cat{E}$ 的一个态射. 我们把笛卡尔积范畴中的一个态射定义为一对态射,
这对态射把一对物件映射到另一对物件. 这对态射可以这样组装起来:
\[(f, g) \circ (f', g') = (f \circ f', g \circ g')\]
这个组合服从结合律, 而且有单位元素 --- 一对恒等态射 $(\id, \id)$. 所以范畴的笛卡尔积确实是一个范畴.

简单理解双函子, 可以把它们看成分别作用在两个参数上的函子. 所以, 与其把函子规则 --- 结合律和单位性质 --- 从函子推广到双函子,
不如对每个参数分别检查是否满足规则. 可惜, 通常分开的函子性不够证明整体的函子性. 整体不满足函子性的范畴叫做 \newterm{premonoidal
 预幺半的}.

在 Haskell 里定义一个双函子. 在这里三个范畴都是一样的: Haskell 类型范畴. 双函子是一个类型构造子, 它接受两个类型参数.
这个 \code{Bifunctor} 类型类的定义直接来自于库 \code{Control.Bifunctor}:

\src{snippet01}

\begin{figure}[H]
  \centering\includegraphics[width=0.3\textwidth]{images/bimap.jpg}
  \caption{bimap}
\end{figure}

类型变量 \code{f} 代表双函子. 它在所有类型签名里都应用到两个类型参数上. 第一个类型签名定义了 \code{bimap}:
一次映射两个函数. 结果是一个抬升的函数 \code{(f a b -> f c d)}, 作用在双函子的类型构造子生成的类型上. \code{bimap}\
有个默认实现, 用 \code{first} 和 \code{second}. (正如前面提到的, 这并不总是可行的, 因为两个映射可能不可交换, 也就是说
\code{first g . second h} 可能不等于 \code{second h . first g}.)

\noindent
另外两个类型签名, \code{first} 和 \code{second}, 是 \code{fmap} 的变体, 它们分别作用在双函子的第一个和第二个参数上,
目睹 \code{f} 在这两个参数上的函子性.

\begin{figure}[H]
  \centering
  \begin{minipage}{0.45\textwidth}
    \centering
    \includegraphics[width=0.65\textwidth]{images/first.jpg} % first figure itself
    \caption{first}
  \end{minipage}\hfill
  \begin{minipage}{0.45\textwidth}
    \centering
    \includegraphics[width=0.6\textwidth]{images/second.jpg} % second figure itself
    \caption{second}
  \end{minipage}
\end{figure}

\noindent
这个类型类定义提供了它们两个的默认实现, 都用 \code{bimap} 来实现.

声明一个 \code{bifunctor} 实例的时候, 你可以选择实现 \code{bimap}, 并接受 \code{first} 和 \code{second} 的默认实现,
或者你也可以只实现 \code{first} 和 \code{second}, 并接受 \code{bimap} 的默认实现 (当然, 你也可以三个都实现, 但是你得
确保它们是正确联系起来的).

\section{双函子的积和余积}

双函子的一个重要例子是范畴积 --- 一对物件的积, 它是通过一个\hyperref[products-and-coproducts-cn]{通用构造}定义出来的.
如果任意一对物件的积都存在, 那么从这些物件到积的映射就是双函子. 这在一般情况下都是成立的, 特别是在 Haskell 里.
这是对构造子的 \code{Bifunctor} 实例 --- 最简单的积类型:

\src{snippet02}[b]
没什么其它的选择: \code{bimap} 简单地把第一个函数应用到一对中的第一个分量, 第二个函数应用到第二个分量. 代码很简单:

\src{snippet03}
这里双函子的行为就是做出类型的对, 例如:

\begin{snip}{haskell}
(,) a b = (a, b)
\end{snip}
我们来说说对称的另一半. 如果余积在范畴中每对物件上都定义了, 那么它也是双函子. 在 Haskell 中, 这表现为 \code{Either} 类型构造子
是 \code{Bifunctor} 的一个实例:

\src{snippet04}[b]
这段代码也不言自明.

现在, 记得我们之前谈过幺半范畴吗? 幺半范畴定义了物件上的一个二元运算符, 还有一个单位物件. 我提过 $\Set$ 是一个幺半范畴,
其运算符是笛卡尔积, 单例集是单位元. 这也是一个在不交并运算下的幺半范畴, 其单位元是空集. 我当时没说幺半范畴要求这个二元
运算符是双函子. 这是个非常重要的要求 --- 我们想要幺半群积与范畴的结构相容, 这个结构由态射所定义. 我们现在离幺半范畴的
定义近了一步(我们还要学习自然性, 才能到那里).

\section{代数数据类型也是函子}

我们看了几个参数化的数据类型, 它们是函子 --- 我们可以为它们定义 \code{fmap}. 复杂的数据类型由简单的数据类型建造.
具体来说, 代数数据类型(\acronym{ADT}s)是通过和与积建造的. 我们刚看到过积和余积都是函子. 我们也知道函子可以组合.
所以如果我们能说明\acronym{ADT}的基本建造单位是函子, 我们就能知道参数化的\acronym{ADT}也是函子.

所以说, 参数化的代数数据类型的基本块是什么? 首先, 有些东西不依赖于函子的类型参数, 例如 \code{Nothing} 在 \code{Maybe}
里, 或者 \code{Nil} 在 \code{List} 里. 它们等价于 \code{Const} 函子. 记得 \code{Const} 函子忽略了它的类型参数
(其实忽略的是 \emph{第二个} 参数, 这是对我们有用的部分. 第一个参数就只是保持不变而已).

接着有一些元素只是把类型参数包装起来, 像是 \code{Just} 在 \code{Maybe} 里的样子. 它们等价于单位函子. 我之前提过单位函子,
作为 \emph{Cat} 范畴的单位态射, 但是没有给出它在 Haskell 里的定义. 它是:

\src{snippet05}

\src{snippet06}
你可以认为 \code{Identity} 是一个最简单的容器, 它总是只包含一个(不可变的)类型为 \code{a} 的值.

代数数据结构中剩下的东西都是从这些原语出发, 用和与积建造出来的.

用刚学的知识, 我们可以重新审视 \code{Maybe} 类型构造子:

\src{snippet07}
这是两个类型之和, 而且我们知道和是函子的. 第一部分 \code{Nothing} 可以表示为 \code{Const ()} 作用在 \code{a} 上
(\code{Const} 的第一个类型参数被设为单位 --- 后面我们会看到 \code{Const} 的更有趣的用法). 第二部分就是单位函子的另一个名字.
我们可以把 \code{Maybe} 在同构意义上定义为:

\src{snippet08}
所以 \code{Maybe} 是 \code{Either} 双函子和两个函子 \code{Const ()} 和 \code{Identity} 的组合.
(\code{Const} 其实是个双函子, 但是这里我们总是用已经部分应用了的它.)

我们已经看到了函子的组合是函子 --- 很容易说服自己对双函子这个结论也是成立的. 我们只需要弄清楚双函子的组合怎么作用在态射上.
给定两个态射, 我们把一个用一个函子抬升, 用另一个函子抬升另一个态射. 我们再用双函子抬升这对抬升后的态射.

我们可以在 Haskell 里表达这种组合. 来定义一个数据类型, 其由一个双函子 \code{bf} (它是一个类型变量, 也是一个类型构造子,
它接受两个类型作为参数), 两个函子 \code{fu} 和 \code{gu} (它们也是类型构造子, 每个接受一个类型参数), 还有两个普通类型
\code{a} 和 \code{b} 参数化. 我们把 \code{fu} 应用到 \code{a}, \code{gu} 应用到 \code{b}, 然后把 \code{bf} 应用到
得到的两个类型上:

\src{snippet09}
这就是物件, 或者说类型的组合. 注意在 Haskell 中我们把类型构造子应用到类型上, 就像我们把函数应用到参数上一样.
词法是相同的.

如果你有点迷失, 试试把 \code{BiComp} 应用到 \code{Either}, \code{Const ()}, \code{Identity}, \code{a}, 和
\code{b} 上, 按这个顺序. 你会得到我们最原始的 \code{Maybe b} (\code{a} 被忽略了).

新的数据类型 \code{BiComp} 是对 \code{a} 和 \code{b} 的双函子, 但是只有当 \code{bf} 本身是双函子, 而且 \code{fu}
和 \code{gu} 都是函子的时候. 编译器必须知道 \code{bf} 有一个 \code{bimap} 的定义, 还有 \code{fu} 和 \code{gu} 都有
\code{fmap} 的定义. 在 Haskell 里, 这是在实例声明里的前提条件: 一系列的类约束, 后面跟着一个双箭头:

\src{snippet10}[b]
\code{BiComp} 的 \code{bimap} 是用 \code{bf} 的 \code{bimap} 和 \code{fu} 和 \code{gu} 的两个 \code{fmap}
实现的. 编译器自动推断出所有的类型, 并且在使用 \code{BiComp} 的时候自动挑选正确的重载函数.

\code{bimap} 定义中的 \code{x} 的类型是:

\src{snippet11}
这够人喝一壶的. 外面的 \code{bimap} 穿过外面的 \code{bf} 层, 两个 \code{fmap} 分别挖进 \code{fu} 和 \code{gu}.
如果 \code{f1} 和 \code{f2} 的类型是:

\src{snippet12}
最终结果的类型就是 \code{bf (fu a') (gu b')}:

\src{snippet13}[b]
如果你喜欢拼图, 这种类型体操可以让你快乐好几个小时了.

所以看来我们不必证明 \code{Maybe} 是函子 --- 因为它是两个有函子性的原语组合, 所以直接就能得出.

善于思考的读者也许会问: 如果代数数据类型的 \code{Functor} 实例那么容易衍生出来, 编译器不能自动执行吗?
确实可以, 已经这样了, 你只需要在源代码的开头加一行 Haskell 扩展:

\begin{snip}{haskell}
{-# LANGUAGE DeriveFunctor #-}
\end{snip}
然后在你的数据结构中加入 \code{deriving Functor}:

\begin{snip}{haskell}
data Maybe a = Nothing | Just a deriving Functor
\end{snip}
相应的 \code{fmap} 就会自动为你生成.

代数数据类型很规矩, 这让它不仅可以自动派生 \code{Functor}, 也可以派生其它的类型类, 包括 \code{Eq} 类型类.
还可以教编译器怎么派生你自己的类型类, 但这有点太难了. 不过其道一以贯之: 你定义基本块还有和与积的行为, 然后让
编译器自动推导出其它的东西.

\section{C++中的函子}

如果你是 C++ 程序员, 就得自己写函子的所有实现. 不过, 你应该能识别出 C++ 中的一些代数数据结构类型. 如果这些
数据结构成了泛型模板, 你应该很容易为它们实现 \code{fmap}.

看一个树的数据结构, 我们会在 Haskell 里把它实现为一个递归的和类型:

\src{snippet14}
正如前面提到的, 可以用一个类的层级在 C++ 实现和类型. 在面向对象的语言里, 自然会想到把 \code{fmap} 实现为基类
\code{Functor} 的虚函数, 然后在所有子类里重载它. 不幸的是, 这是不可能的, 因为 \code{fmap} 是一个模板, 它的参数
不仅是其作用的对象的类型 (\code{this} 指针), 还有应用的函数的返回类型. 虚函数不能是模板. 我们会把 \code{fmap}
实现为一个泛型的自由函数, 然后用 \code{dynamic\_cast} 替代模式匹配.

基类必须至少定义一个虚函数, 这样才能支持动态转换, 所以我们把析构函数定义为虚函数(这也是个好习惯):

\begin{snip}{cpp}
template<class T>
struct Tree {
    virtual ~Tree() {}
};
\end{snip}
\code{Leaf} 只是 \code{Identity} 函子的伪装:

\begin{snip}{cpp}
template<class T>
struct Leaf : public Tree<T> {
    T _label;
    Leaf(T l) : _label(l) {}
};
\end{snip}
\code{Node} 是一个积类型:

\begin{snip}{cpp}
template<class T>
struct Node : public Tree<T> {
    Tree<T> * _left;
    Tree<T> * _right;
    Node(Tree<T> * l, Tree<T> * r) : _left(l), _right(r) {}
};
\end{snip}
实现 \code{fmap} 的时候, 我们利用了对 \code{Tree} 类型的动态分派. \code{Leaf} 的情况下, 我们应用 \code{fmap} 的
\code{Identity} 版本, \code{Node} 的情况下像处理两个函子组合成的双函子. 作为 C++ 程序员, 你可能不习惯这样分析代码,
但这可以训练范畴论思维.

\begin{snip}{cpp}
template<class A, class B>
Tree<B> * fmap(std::function<B(A)> f, Tree<A> * t) {
    Leaf<A> * pl = dynamic_cast <Leaf<A>*>(t);
    if (pl)
        return new Leaf<B>(f (pl->_label));
    Node<A> * pn = dynamic_cast<Node<A>*>(t);
    if (pn)
        return new Node<B>( fmap<A>(f, pn->_left)
                          , fmap<A>(f, pn->_right));
    return nullptr;
}
\end{snip}
为了简单, 我忽略了内存和资源管理, 但在生产代码中你通常会用智能指针(根据你的策略, 可能是 \code{unique} 或者
\code{shared} 指针).

对比下 Haskell 实现的 \code{fmap}:

\src{snippet15}
这个实现也可以由编译器自动生成.

\section{写者函子}

我承诺过要回到 \hyperref[kleisli-categories-zh-cn]{Kleisli 范畴}. 这个范畴里的态射表示为 ``装饰'' 函数,
返回 \code{Writer} 数据结构.

\src{snippet16}
我说过这装饰和自函子有关. 而且, 确实, \code{Writer} 类型构造子是对 \code{a} 来说是具有函子性的. 我们甚至不需要
实现 \code{fmap}, 因为它只是个简单的积类型.

但是 Kleisli 范畴同函子之间的关系是什么 --- 一般情况下? Kleisli 范畴是一个范畴, 所以它定义了组合和单位元素.
让我提醒你一下, 组合用的是鱼操作符:

\src{snippet17}
单位态射是 \code{return} 函数:

\src{snippet18}
如果你看这两个函数的类型很久(我意思是, 真的很\emph{久}), 你可以把它们组合起来, 得到一个函数, 它的类型签名
可以作为 \code{fmap} 的类型签名. 就像这样:

\src{snippet19}
这里, 鱼运算符组合了两个函数: 一个是熟悉的 \code{id}, 另一个是一个 lambda, 它在 return 后面写了个(f x).
最难绕过来的也许是 \code{id}. 鱼运算符的参数应该是一个函数, 它接受一个``正常''类型, 返回一个装饰过的类型, 不是
吗? 嗯... 不完全是, 没人说 \code{a -> Writer b} 里的 \code{a} 必须是``正常''类型. 它是个类型变量, 所以什么
都行, 在这就是一个装饰过的类型, 像 \code{Writer b}.

所以 \code{id} 会把 \code{Writer a} 变成 \code{Writer a}. 鱼运算符会把 \code{a} 的值取出来, 传给 lambda.
在那里, \code{f} 会把它变成 \code{b}, \code{return} 会把它装饰起来, 变成 \code{Writer b}. 把这些组合起来,
我们就得到了 \code{fmap} 理应得到的结果.

注意这个论证是非常一般的: 你可以把 \code{Writer} 替换成任何类型构造子. 只要它支持鱼运算符和 \code{return},
你就可以定义 \code{fmap}. 所以 Kleisli 范畴里的装饰总是一个函子. (不过不是每个函子都能产生 Kleisli 范畴.)

你也许会好奇, 我们刚刚定义的 \code{fmap} 是不是编译器会为我们自动生成的 \code{deriving Functor}. 很有意思
, 正是如此. 这是因为 Haskell 怎么实现多态函数的方式. 它叫做 \newterm{parametric polymorphism 参数多态}
, 是所谓 \newterm{theorems for free 自由定理}的源头. 其中一个定理说, 任给类型构造子, 如果有一个保持单位性质的
\code{fmap} 实现, 那么它必须是唯一的.

\section{协变和逆变函子}

既然我们复习了写者函子, 可以回到读者函子. 它是基于部分应用的函数箭头类型构造子:

\src{snippet20}
我们可以重写它为一个同义的类型:

\src{snippet21}
\code{Functor} 的实例, 就像我们之前看到的, 是:

\src{snippet22}
但就像对类型构造子, 或者 \code{Either} 类型构造子, 这个函数类型构造子也有两个类型参数. 对和 \code{Either} 在
两个参数上都有函子性 --- 它们是双函子. 函数构造子也是双函子吗?

试试让它的第一个参数是函子. 我们从一个同义类型开始 --- 就像 \code{Reader} 一样, 只是参数的顺序反过来了:

\src{snippet23}
这回我们固定返回类型 \code{r}, 变化参数类型 \code{a}. 让我们看看能不能匹配类型, 以便实现 \code{fmap}, 它的类型签名是:

\src{snippet24}
用两个接受 \code{a}, 却分别返回 \code{b} 和 \code{r} 的函数, 我们根本得不到一个接受 \code{b} 返回 \code{r} 的函数!
如果我们能反转第一个函数, 使它接受 \code{b} 返回 \code{a}, 就可以了. 我们不能反转任意函数, 但是我们可以去到反过来的范畴.

回顾: 对于任何范畴 $\cat{C}$, 有一个对偶范畴 $\cat{C}^{op}$. 它有相同的物件, 但是所有的箭头都反过来了.

考虑一个从 $\cat{C}^{op}$ 到另一个范畴 $\cat{D}$ 的函子:
\[F \Colon \cat{C}^{op} \to \cat{D}\]
这个函子把 $\cat{C}^{op}$ 里的态射 $f^{op} \Colon a \to b$ 映射到 $\cat{D}$ 里的态射 $F f^{op} \Colon F a \to F b$.
但是态射 $f^{op}$ 秘密地对应着原始范畴 $\cat{C}$ 里的态射 $f \Colon b \to a$. 注意反转.

现在, $F$ 是一个常规的函子, 但是我们可以基于 $F$ 定义另一个映射, 它不是函子 --- 让我们叫它 $G$. 它是从 $\cat{C}$ 到
$\cat{D}$ 的映射. 它映射物件的方式和 $F$ 一样, 但是当映射态射的时候, 它反转它们. 它接受范畴 $\cat{C}$ 里的态射
$f \Colon b \to a$, 先把它反转成 $f^{op} \Colon a \to b$, 然后用函子 $F$ 映射, 得到 $F f^{op} \Colon F a \to F b$.

考虑到 $F a$ 和 $G a$ 相同, $F b$ 和 $G b$ 相同, 整个过程可以描述为: $G f \Colon (b \to a) \to (G a \to G b)$
这是一个 ``扭曲的函子''. \emph{逆变函子} 用来称呼像这样把态射的方向反转的范畴映射. 留意, 逆变函子只是反范畴里的常规函子.
顺带一提, 常规函子 --- 我们一直在学习的函子 --- 称为 \emph{协变函子}.

\begin{figure}[H]
  \centering
  \includegraphics[width=40mm]{images/contravariant.jpg}
\end{figure}

\noindent
这是 Haskell 里定义的逆变函子类型类(实际上是逆变的 \emph{自}函子):

\src{snippet25}
我们的类型构造子 \code{Op} 是它的一个实例:

\src{snippet26}
留意函数 \code{f} 把它自己插到(就是说放到右边)\code{Op} 的内容 --- 函数 \code{g} --- 之前.

\code{Op} 的 \code{contramap} 定义可以更简洁, 如果你注意到它就是函数组合运算符, 只是参数反过来了. 有个专门的函数
用来翻转参数, 叫做 \code{flip}:

\src{snippet27}
用它, 我们得到:

\src{snippet28}

\section{前函子}

我们看到函数箭头操作符的第一个参数是逆变的, 第二个参数是协变的. 这种巨兽有名字吗? 如果目标范畴是 $\Set$, 这种巨兽
叫做 \newterm{profunctor 前函子}. 因为逆变函子等价于反范畴的协变函子, 前函子可以定义为:
\[\cat{C}^{op} \times \cat{D} \to \Set\]
因为 Haskell 的类型就是集合(先近似), 我们把 \code{Profunctor} 这个名字给一个有两个参数的类型构造子 \code{p}.
它的第一个参数是逆变的, 第二个参数是协变的. 这是从 \code{Data.Profunctor} 库里取出的类型类:

\src{snippet29}[b]
所有这三个函数都有默认实现. 就像 \code{Bifunctor} 一样, 当声明 \code{Profunctor} 的实例的时候, 你可以选择实现
\code{dimap}, 并接受 \code{lmap} 和 \code{rmap} 的默认实现, 或者你也可以只实现 \code{lmap} 和 \code{rmap},
并接受 \code{dimap} 的默认实现.

\begin{figure}[H]
  \centering
  \includegraphics[width=0.4\textwidth]{images/dimap.jpg}
  \caption{dimap}
\end{figure}

\noindent
现在我们可以断言函数箭头操作符是 \code{Profunctor} 的一个实例:

\src{snippet30}[b]
前函子应用在 Haskell 的 lens 库中. 讨论端和余端的时候我们会再次看到它们.

\section{Hom-函子}

上面的例子都反映了一个下面这个更一般的陈述. 这个映射是函子: 接受一对物件 $a$ 和 $b$, 把这个对映射到它们之间的态射
集合 $\cat{C}(a, b)$. 这个函子从积范畴 $\cat{C}^{op} \times \cat{C}$ 到集合范畴 $\Set$.

来定义它在态射上的行为. 在 $\cat{C}^{op} \times \cat{C}$ 里, 一个态射是来自 $\cat{C}$ 的一对态射:
\begin{gather*}
  f \Colon a' \to a \\
  g \Colon b \to b'
\end{gather*}
这个对抬升后的结果必须是一个从集合 $\cat{C}(a, b)$ 到集合 $\cat{C}(a', b')$ 的态射(函数). 只需挑出
$\cat{C}(a, b)$ 里的任意元素 $h$(它是一个从 $a$ 到 $b$ 的态射), 然后给它:
\[g \circ h \circ f\]
这是 $\cat{C}(a', b')$ 中的一个元素.

如你所见, hom-函子是前函子的特例.

\section{挑战}

\begin{enumerate}
  \tightlist
  \item
        说明数据类型:

        \begin{snip}{haskell}
data Pair a b = Pair a b
\end{snip}

        是一个双函子. 加分项: 实现 \code{Bifunctor} 的全部三个方法, 然后用等式推理说明这些定义
        和默认实现是兼容的.
  \item
        说明 \code{Maybe} 的标准定义和这个苦涩的定义是同构的:

        \begin{snip}{haskell}
type Maybe' a = Either (Const () a) (Identity a)
\end{snip}

        提示: 定义这两种实现的两个映射. 加分项: 用等式推理说明它们是互逆的.
  \item
        试试另外一种数据结构. 我叫它 \code{PreList}, 因为它是 \code{List} 的前身. 它用一个类型参数
        \code{b} 替换了递归.

        \begin{snip}{haskell}
data PreList a b = Nil | Cons a b
\end{snip}

        你可以轻易回复我们之前 \code{List} 的定义. 只要递归地应用 \code{PreList} 到自己(我们会在
        谈到不动点的时候说说怎么做).

        说明 \code{PreList} 是 \code{Bifunctor} 的实例.
  \item
        说明下面的数据类型在 \code{a} 和 \code{b} 里定义了双函子:

        \begin{snip}{haskell}
data K2 c a b = K2 c

data Fst a b = Fst a

data Snd a b = Snd b
\end{snip}

        加分项: 对比你的解和 Conor McBride 的论文 \urlref{http://strictlypositive.org/CJ.pdf}
        {Clowns to the Left of me, Jokers to the Right}.
  \item
        用不是 Haskell 的语言定义一个双函子. 为那个语言的泛型对实现 \code{bimap}.
  \item
        \code{std::map} 可以被看作在模板参数 \code{Key} 和 \code{T} 上的双函子或者前函子吗?
        你怎么重新设计这个结构, 让它可以呢?
\end{enumerate}
