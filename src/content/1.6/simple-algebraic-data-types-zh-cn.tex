% !TEX root = ../../ctfp-print.tex

\lettrine[lhang=0.17]{学}{过两种}组合类型的方法: 积和余积. 日常编程中我们可以用这两种
方法建造出大量的数据结构. 这有重要的实践意义. 许多数据结构的性质是可组合的. 例如, 如果你知道
如何判等基本数据类型, 你就可以把比较推广到积类型和余积类型里来. 在 Haskell 你可以
为大量组合类型自动衍生判等, 比较, 字符串转换等更多功能.

我们仔细看看编程中的积类型与和类型.

\section{积类型}

编程语言里, 两类型积的经典实现就是对. 在 Haskell 里, 对是一个原始类型构造子; 在 C++ 里, 它是
标准库里一个相当复杂的模板.

\begin{figure}[H]
  \centering
  \includegraphics[width=0.35\textwidth]{images/pair.jpg}
\end{figure}

\noindent
对不是严格可交换的: 一个对 \code{(Int, Bool)} 不能替换另一个对 \code{(Bool, Int)}, 即使它们
有相同的信息. 不过在同构意义上, 它们是可交换的. \code{swap} 函数给出这个同构(它是自己的逆):

\src{snippet01}
你可以认为这两个对不过用不同的格式存储同样的信息. 就像大端序和小端序一样.

可以把任意数量的类型组合成积类型, 只要把对嵌套在对里. 但是有更简单的方法: 嵌套对是等价于元组的.
因为其实不同的嵌套方式是同构的. 如果你想把三个类型 \code{a}, \code{b}, 和 \code{c} 按刚才的顺序
组合成积, 可以有两种做法:

\src{snippet02}

或者
\src{snippet03}
这些类型不同 --- 你不能把一个传给需要另一个的函数 --- 但是它们的元素一一对应. 有一个函数把它们
映射起来:

\src{snippet04}
这个函数是可逆的:

\src{snippet05}
所以这是一个同构. 同样的信息, 只是打包方式不同.

你可以把创建积类型的过程看作是类型的二元运算. 从这个角度看, 上面的同构就像幺半群的结合律:
\[(a * b) * c = a * (b * c)\]
除了在幺半群的情况下, 两种组合积的方式就是相等的; 而在这里只是 ``同构意义上'' 相等的.

如果我们只看同构, 不去管严格相等, 我们可以更进一步, 说明单元类型 \code{()} 是积类型的单位元,
就像 1 是乘法的单位元. 确实, 一个值和单元类型配对不会增加任何信息. 这类型:

\src{snippet06}
与 \code{a} 同构. 下面是同构:

\src{snippet07}

\src{snippet08}
这些观察可以形式化为说 $\Set$ (集合范畴) 是一个\newterm{monidal category 幺半范畴}. 它是一个范畴,
也是一个幺半群, 因为你可以把对象相乘 (这里是取笛卡尔积). 我会在后面讲更多关于幺半范畴的内容, 并给出完整定义.

Haskell 中定义积类型还有一种更一般的方法 --- 特别是当它们和和类型结合时. 它用多参数的具名构造子. 例如对
也可以这样定义:

\src{snippet09}
这里 \code{Pair a b} 是一个类型的名字, 这类型用两个其他类型 \code{a} 和 \code{b} 参数化; \code{P}
是数据构造子的名字. 你把两个类型传给 \code{Pair} 类型构造子来定义一个对类型. 你把两个恰当类型的值传给
构造子 \code{P} 来构造一个对值. 例如, 我们定义一个值 \code{stmt} 为 \code{String} 和 \code{Bool} 的对:

\src{snippet10}
第一行是类型声明. 它在泛型的类型构造子 \code{Pair} 的定义中, 用 \code{String} 和 \code{Bool} 替换 \code{a} 和
\code{b}. 第二行定义了实际的值, 把一个具体的布尔和一个具体的字符串给数据构造子 \code{P}. 类型构造子用来构造类型;
数据构造子用来构造值.

因为类型构造子和数据构造子的命名空间是分开的, 你会经常看到同样的名字用在两者上, 就像:

\src{snippet11}
如果你仔细看, 你会发现内置的对类型就是这种声明的一种变体, 那里 \code{Pair} 换成了二元运算符 \code{(,)}.
其实你可以像其他具名的构造子一样用 \code{(,)} 来创建对, 用前缀表示法:

\src{snippet12}
同样的, 你可以用 \code{(,,)} 来创建三元组, 以此类推.

除了用泛型的对或者元组, 你也可以定义具名的积类型, 就像:

\src{snippet13}
这就是 \code{String} 和 \code{Bool} 之积, 但它有自己的名字和构造子. 这种风格的优势在于你可以定义很多类型, 它们有
相同的类型, 但是意思和功能不同, 所以不能互相替换.

编程用元组和多参数构造子可能会混乱和出错 --- 你要记住每个分量代表什么. 用具名的分量名字会更好. 一个有具名分量的积类型
在 Haskell 里叫\newterm{record (记录)}, 在 C 里叫\code{struct (结构)}.

\section{记录}

看一个简单的例子. 想要用一个数据结构表示化学元素, 其中有两个字符串---名称和符号; 还有一个整数代表原子序数. 可以用
元组 \code{(String, String, Int)} 来表示, 但是要记住每个分量代表什么. 用模式匹配提取分量, 像这个函数里检查元素
的符号是否为其名称的前缀(比如 \textbf{He} 是 \textbf{Helium} 的前缀):

\src{snippet14}
这段代码易错, 而且不可读不可维护. 定义一个记录会好很多:

\src{snippet15}
这两种表示是同构的, 有这两个转换函数, 它们是互逆的:

\src{snippet16}

\src{snippet17}
注意记录的名字同时是访问器函数. 例如, \code{atomicNumber e} 从 \code{e} 中取出 \code{atomicNumber} 字段.
我们把 \code{atomicNumber} 当作这个类型的函数:

\src{snippet18}
用 \code{Element} 的记录句法, 我们的函数 \code{startsWithSymbol} 变得更可读:

\src{snippet19}
我们甚至可以用 Haskell 的技巧, 把函数 \code{isPrefixOf} 变成中缀运算符, 使它读起来像句子:

\src{snippet20}
这里的括号可以省略, 因为中缀运算符的优先级比函数调用低.

\section{和类型}

正如集合范畴里的积产生了积类型, 余积运算给出了和类型. Haskell 中和类型的经典实现是:

\src{snippet21}
就像对, \code{Either} 也是可交换的(同构意义上), 可以嵌套, 嵌套顺序无关(同构意义上).
所以我们可以定义三元的和类型:

\src{snippet22}
以此类推.

在余积运算下, $Set$ 也是一个(对称的)幺半群. 二元运算是不交并, 单位元是初始物件. 用类型来说, \code{Either} 是
幺半群的运算符, \code{Void} 是单位元. 你可以把 \code{Either} 看作加法, \code{Void} 看作零. 确实, 把 \code{Void}
加到任何和类型中都不会改变它的值. 例如:

\src{snippet23}
和 \code{a} 同构. 因为不能构造出这个类型的 \code{Right} 部分,  --- 没有 \code{Void} 类型的值. 所以类型
\code{Either a Void} 只能用 \code{Left} 构造出来, 它只是 \code{a} 的包装. 所以符号化为 $a + 0 = a$.

和类型在 Haskell 中很常见, 但它们在 C++ 中没那么常用. 有很多原因.

首先, 最简单的和类型就是枚举, 在 C++ 里用 \code{enum} 实现. Haskll 中的这个和类型:

\src{snippet24}
是 C++ 中的:

\begin{snip}{cpp}
enum { Red, Green, Blue };
\end{snip}
一个更简单的和类型:

\src{snippet25}
是 C++ 中原始类型 \code{bool}.

有时需要用和类型来表达"缺少值". 这个简单的和类型在 C++ 中一般用特殊的技巧或者所谓的 ``不可能值'' 实现, 例如
空字符串, 负数, 空指针等等. 如果真要表达这种可选性, 在 Haskell 中用 \code{Maybe} 类型:

\src{snippet26}
\code{Maybe} 类型就是两个类型的和. 你可以把两个构造子分开, 这样就能看得更清楚了. 第一个是这样:

\src{snippet27}
这是只有一个值的枚举, 叫做 \code{Nothing}. 换句话说, 这是个单例, 也就等价于单位类型 \code{()}.
第二个部分:

\src{snippet28}
就是一个 \code{a} 类型的封装. 我们可以这样编码 \code{Maybe}:

\src{snippet29}
更复杂的和类型在 C++ 中通常用指针来伪装. 指针可以是空的, 或者指向某个值. 例如, Haskell 中的列表类型可以定义成
一个递归的和类型:

\src{snippet30}
翻译成 C++ 的时候可以用空指针来实现空列表:

\begin{snip}{cpp}
template<class A>
class List {
    Node<A> * _head;
public:
    List() : _head(nullptr) {} // Nil
    List(A a, List<A> l)       // Cons
      : _head(new Node<A>(a, l))
    {}
};
\end{snip}
注意 Haskell 的两个构造子 \code{Nil} 和 \code{Cons} 在 C++ 中变成了两个重载的 \code{List} 构造函数.
\code{List} 类不需要标签来区分和类型的两个部分. 它让 \code{\_head} 是特殊的 \code{nullptr} 的值来代表
\code{Nil}.

不过 C++ 和 Haskell 类型最重要的差别还是 Haskell 中的类型不可变. 如果你用一个构造子创建了一个对象, 这对象会
永远记住构造它时候的构造子和参数. 所以用 \code{Just "energy"} 创建的 \code{Maybe} 对象永远不会变成 \code{Nothing}.
类似, 空列表永远是空的, 三个元素的列表永远都是这三个元素的列表.

这种不变性让构造是可逆的. 给一个对象, 总可以把它分拆成构造它用的部分. 这个分拆用模式匹配实现, 它重用构造子作为模式.
如果构造子有参数的话, 就用变量(或者其它模式)来代替.

\code{List} 数据类型有两个构造子, 所以解构 \code{List} 用两个模式来匹配. 一个匹配空列表 \code{Nil}, 另一个匹配
\code{Cons} 构造的列表. 例如, 这个简单的 \code{List} 函数:

\src{snippet31}
\code{maybeTail} 定义的第一部分用 \code{Nil} 构造子作为模式, 返回 \code{Nothing}. 第二部分用 \code{Cons} 构造子作为模式.
它把第一个构造子参数换成通配符, 因为我们不关心它. 第二个参数绑定到变量 \code{t} (我会把这些东西叫做变量, 尽管严格来说,
它们从不变化: 一旦绑定到一个表达式, 变量就不会再变了). 返回值是 \code{Just t}. 现在, 根据你的 \code{List} 是怎么创建的,
它会匹配其中一个模式. 如果它是用 \code{Cons} 创建的, 传给它的两个参数就会被取出来(第一个被丢弃).

用多态的类层级, 可以在 C++ 中实现更复杂的和类型. 一族共祖的类可以看作是一个和类型, 其中虚函数表充当隐藏的标签.
Haskell 中用模式匹配做的事情, C++ 中用虚函数指针分派相应虚函数来做.

你很少会在 C++ 中看到用 \code{union} 做和类型. 因为 \code{union} 有严格的限制, 你甚至不能把 \code{std::string}
放进联合里, 因为它有个拷贝构造函数.

\section{类型的代数}

积类型与和类型中任选一个就能定义出大量有用的数据结构, 但更强大的是它们二者一起用. 再一次, 我们在使用组合的力量.

总结一下我们说过的东西. 我们看到类型系统下有两个可交换的幺半群: 一个是和类型, 单位元是 \code{Void}; 另一个是积类型,
单位元是单元类型 \code{()}. 我们想把它们看作加法和乘法. 在这个类比下, \code{Void} 对应于零, 单元类型 \code{()}
对应于一.

看看怎么继续这个比喻. 例如, 乘以零会得到零吗? 换句话说, 如果一个积类型有个分量是 \code{Void}, 它是不是同构于 \code{Void}?
例如, 能不能构造一个对, 其类型为 \code{Void} 和 \code{Int}?

要有两个值才能构造出一个对. 虽然你可以随便拿出一个整数, 但没有 \code{Void} 类型的值. 所以对于任何类型 \code{a},
类型 \code{(a, Void)} 是不可构造的 --- 没有值. 所以它和 \code{Void} 等价. 换句话说, $a \times 0 = 0$.

另外, 乘法和加法有分配律: \[a \times (b + c) = a \times b + a \times c\]
积类型与和类型也有这个性质吗? 是的, --- 同构意义上. 左边对应的类型是:

\src{snippet32}
右边对应的类型是:

\src{snippet33}
这是转换它们的函数:

\src{snippet34}
转换的另一边:

\src{snippet35}
\code{case of} 语句用来在函数里模式匹配. 每个模式后面跟一个箭头和一个表达式, 当模式匹配时, 表达式就会被求值.
例如, 如果你这样调用 \code{prodToSum}:

\src{snippet36}
\code{case e of} 中的 \code{e} 会等于 \code{Left "Hi!"}. 它会匹配模式 \code{Left y}, 把 \code{"Hi!"} 替换到 \code{y}.
因为 \code{x} 已经匹配到了 \code{2}, \code{case of} 语句的结果, 也就是整个函数的结果, 会是 \code{Left (2, "Hi!")},
正如所料.

我不打算证明这两个函数互逆, 你想想就知道是一定的. 它们只是简单重组两个数据结构的内容. 相同的数据, 不同的格式.

数学家给这种两个缠绕起来的幺半群结构取了一个名字: \newterm{semiring 半环}. 它不是一个完整的\newterm{ring 环}, 因为
我们不能定义类型的减法. 这就是为什么半环有时候叫做\newterm{rig}, 这是 ``ring 没有 \emph{n}(negative)'' 的谐音.
不过无论如何, 我们可以把其他半环(比如自然数)上的东西翻译到类型里来, 这样得到不少启发. 下面是一个翻译表, 其中有一些
有意义的条目:

\begin{longtable}[]{@{}ll@{}}
  \toprule
  数字 & 类型\tabularnewline
  \midrule
  \endhead
  $0$          & \code{Void}\tabularnewline
  $1$          & \code{()}\tabularnewline
  $a + b$      & \code{Either a b = Left a | Right b}\tabularnewline
  $a \times b$ & \code{(a, b)} or \code{Pair a b = Pair a b}\tabularnewline
  $2 = 1 + 1$  & \code{data Bool = True | False}\tabularnewline
  $1 + a$      & \code{data Maybe = Nothing | Just a}\tabularnewline
  \bottomrule
\end{longtable}

\noindent
列表类型很有趣, 因为它定义成一个方程的解. 我们要定义的类型出现在等式的两边:

\src{snippet37}
如果做往常的替换, 而且还把 \code{List a} 替换成 \code{x}, 我们得到方程:

\begin{Verbatim}
  x = 1 + a * x
\end{Verbatim}
不能用传统代数方法解这个方程, 因为我们没法在类型上做除法和减法. 但是可以试试替换, 在右边用 \code{(1 + a * x)} 替换
\code{x}, 然后用分配律. 这样就得到了一个级数:

\begin{Verbatim}
  x = 1 + a*x
  x = 1 + a*(1 + a*x) = 1 + a + a*a*x
  x = 1 + a + a*a*(1 + a*x) = 1 + a + a*a + a*a*a*x
  ...
  x = 1 + a + a*a + a*a*a + a*a*a*a...
\end{Verbatim}
最后有了一个无穷的积(元组)之和, 这可以解释成: 列表可以是空的, \code{1}; 或者单例, \code{a}; 或者对, \code{a*a};
或者三元组, \code{a*a*a}; 等等\ldots{} 这正是列表的定义 --- 一串 \code{a}!

关于列表还有很多. 学过函子和不动点之后, 我们会回来, 然后讨论更多递归数据结构.

用符号变量解方程 --- 这是代数! 所以这些类型有了这个名字: 代数数据类型.

最后, 我要提到一个类型代数非常重要的一个解读. 注意两个类型 \code{a} 和 \code{b} 的积类型 \code{(a, b)} 必须
包含类型 \code{a} \emph{和} 类型 \code{b} 的值, 所以两个类型都必须是可构造的. 另一方面, 两个类型的和类型
\code{Either a b} 只需要包含类型 \code{a} \emph{或者} 类型 \code{b} 的值, 所以只要其中一个是可构造的就行.
逻辑的 \emph{和} 与 \emph{或} 也是一个半环, 它也可以映射到类型理论中:

\begin{longtable}[]{@{}ll@{}}
  \toprule
  逻辑                & 类型\tabularnewline
  \midrule
  \endhead
  $\mathit{false}$     & \code{Void}\tabularnewline
  $\mathit{true}$      & \code{()}\tabularnewline
  $a \mathbin{||} b$   & \code{Either a b = Left a | Right b}\tabularnewline
  $a \mathbin{\&\&} b$ & \code{(a, b)}\tabularnewline
  \bottomrule
\end{longtable}

\noindent
这个比喻有更深的意味, 这是逻辑与类型理论中 Curry-Howard 同构的基础. 讨论函数类型的时候, 我们会回到这个问题上来.

\section{挑战}

\begin{enumerate}
  \tightlist
  \item
        说明 \code{Maybe a} 和 \code{Either () a} 是同构的.
  \item
        下面是 Haskell 里的一个和类型:

        \begin{snip}{haskell}
data Shape = Circle Float
           | Rect Float Float
\end{snip}
        想在 \code{Shape} 上定义类似 \code{area} 的函数, 需要模式匹配两个构造子:

        \begin{snip}{haskell}
area :: Shape -> Float
area (Circle r) = pi * r * r
area (Rect d h) = d * h
\end{snip}
        在 C++ 或 Jave 里把 \code{Shape} 实现为一个接口, 新建两个类: \code{Circle} 和 \code{Rect}.
        把 \code{area} 实现为一个虚函数.
  \item
        继续前面的例子: 可以轻易增加一个新的函数 \code{circ} 计算 \code{Shape} 的周长.
        这样做不需要动 \code{Shape} 的定义:

        \begin{snip}{haskell}
circ :: Shape -> Float
circ (Circle r) = 2.0 * pi * r
circ (Rect d h) = 2.0 * (d + h)
\end{snip}
        在你 C++ 或者 Java 的实现里, 你需要修改哪些代码来加入它?
  \item
        继续: 加入新的形状 \code{Square} 到 \code{Shape} 里, 并做所有必要的更新.
        你在 Haskell 里需要修改哪些代码, 在 C++ 或者 Java 里呢? (即使你不是 Haskell 程序员,
        修改应该也很明显.)
  \item
        说明 $a + a = 2 \times a$ 对于类型(同构意义上)成立. 回想在我们的翻译表里 $2$ 对应 \code{Bool}.
\end{enumerate}
